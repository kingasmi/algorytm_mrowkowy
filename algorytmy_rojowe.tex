\documentclass[11pt]{article}

    \usepackage[breakable]{tcolorbox}
    \usepackage{parskip} % Stop auto-indenting (to mimic markdown behaviour)
    

    % Basic figure setup, for now with no caption control since it's done
    % automatically by Pandoc (which extracts ![](path) syntax from Markdown).
    \usepackage{graphicx}
    % Maintain compatibility with old templates. Remove in nbconvert 6.0
    \let\Oldincludegraphics\includegraphics
    % Ensure that by default, figures have no caption (until we provide a
    % proper Figure object with a Caption API and a way to capture that
    % in the conversion process - todo).
    \usepackage{caption}
    \DeclareCaptionFormat{nocaption}{}
    \captionsetup{format=nocaption,aboveskip=0pt,belowskip=0pt}

    \usepackage{float}
    \floatplacement{figure}{H} % forces figures to be placed at the correct location
    \usepackage{xcolor} % Allow colors to be defined
    \usepackage{enumerate} % Needed for markdown enumerations to work
    \usepackage{geometry} % Used to adjust the document margins
    \usepackage{amsmath} % Equations
    \usepackage{amssymb} % Equations
    \usepackage{textcomp} % defines textquotesingle
    % Hack from http://tex.stackexchange.com/a/47451/13684:
    \AtBeginDocument{%
        \def\PYZsq{\textquotesingle}% Upright quotes in Pygmentized code
    }
    \usepackage{upquote} % Upright quotes for verbatim code
    \usepackage{eurosym} % defines \euro

    \usepackage{iftex}
    \ifPDFTeX
        \usepackage[T1]{fontenc}
        \IfFileExists{alphabeta.sty}{
              \usepackage{alphabeta}
          }{
              \usepackage[mathletters]{ucs}
              \usepackage[utf8x]{inputenc}
          }
    \else
        \usepackage{fontspec}
        \usepackage{unicode-math}
    \fi

    \usepackage{fancyvrb} % verbatim replacement that allows latex
    \usepackage{grffile} % extends the file name processing of package graphics
                         % to support a larger range
    \makeatletter % fix for old versions of grffile with XeLaTeX
    \@ifpackagelater{grffile}{2019/11/01}
    {
      % Do nothing on new versions
    }
    {
      \def\Gread@@xetex#1{%
        \IfFileExists{"\Gin@base".bb}%
        {\Gread@eps{\Gin@base.bb}}%
        {\Gread@@xetex@aux#1}%
      }
    }
    \makeatother
    \usepackage[Export]{adjustbox} % Used to constrain images to a maximum size
    \adjustboxset{max size={0.9\linewidth}{0.9\paperheight}}

    % The hyperref package gives us a pdf with properly built
    % internal navigation ('pdf bookmarks' for the table of contents,
    % internal cross-reference links, web links for URLs, etc.)
    \usepackage{hyperref}
    % The default LaTeX title has an obnoxious amount of whitespace. By default,
    % titling removes some of it. It also provides customization options.
    \usepackage{titling}
    \usepackage{longtable} % longtable support required by pandoc >1.10
    \usepackage{booktabs}  % table support for pandoc > 1.12.2
    \usepackage{array}     % table support for pandoc >= 2.11.3
    \usepackage{calc}      % table minipage width calculation for pandoc >= 2.11.1
    \usepackage[inline]{enumitem} % IRkernel/repr support (it uses the enumerate* environment)
    \usepackage[normalem]{ulem} % ulem is needed to support strikethroughs (\sout)
                                % normalem makes italics be italics, not underlines
    \usepackage{mathrsfs}
    

    
    % Colors for the hyperref package
    \definecolor{urlcolor}{rgb}{0,.145,.698}
    \definecolor{linkcolor}{rgb}{.71,0.21,0.01}
    \definecolor{citecolor}{rgb}{.12,.54,.11}

    % ANSI colors
    \definecolor{ansi-black}{HTML}{3E424D}
    \definecolor{ansi-black-intense}{HTML}{282C36}
    \definecolor{ansi-red}{HTML}{E75C58}
    \definecolor{ansi-red-intense}{HTML}{B22B31}
    \definecolor{ansi-green}{HTML}{00A250}
    \definecolor{ansi-green-intense}{HTML}{007427}
    \definecolor{ansi-yellow}{HTML}{DDB62B}
    \definecolor{ansi-yellow-intense}{HTML}{B27D12}
    \definecolor{ansi-blue}{HTML}{208FFB}
    \definecolor{ansi-blue-intense}{HTML}{0065CA}
    \definecolor{ansi-magenta}{HTML}{D160C4}
    \definecolor{ansi-magenta-intense}{HTML}{A03196}
    \definecolor{ansi-cyan}{HTML}{60C6C8}
    \definecolor{ansi-cyan-intense}{HTML}{258F8F}
    \definecolor{ansi-white}{HTML}{C5C1B4}
    \definecolor{ansi-white-intense}{HTML}{A1A6B2}
    \definecolor{ansi-default-inverse-fg}{HTML}{FFFFFF}
    \definecolor{ansi-default-inverse-bg}{HTML}{000000}

    % common color for the border for error outputs.
    \definecolor{outerrorbackground}{HTML}{FFDFDF}

    % commands and environments needed by pandoc snippets
    % extracted from the output of `pandoc -s`
    \providecommand{\tightlist}{%
      \setlength{\itemsep}{0pt}\setlength{\parskip}{0pt}}
    \DefineVerbatimEnvironment{Highlighting}{Verbatim}{commandchars=\\\{\}}
    % Add ',fontsize=\small' for more characters per line
    \newenvironment{Shaded}{}{}
    \newcommand{\KeywordTok}[1]{\textcolor[rgb]{0.00,0.44,0.13}{\textbf{{#1}}}}
    \newcommand{\DataTypeTok}[1]{\textcolor[rgb]{0.56,0.13,0.00}{{#1}}}
    \newcommand{\DecValTok}[1]{\textcolor[rgb]{0.25,0.63,0.44}{{#1}}}
    \newcommand{\BaseNTok}[1]{\textcolor[rgb]{0.25,0.63,0.44}{{#1}}}
    \newcommand{\FloatTok}[1]{\textcolor[rgb]{0.25,0.63,0.44}{{#1}}}
    \newcommand{\CharTok}[1]{\textcolor[rgb]{0.25,0.44,0.63}{{#1}}}
    \newcommand{\StringTok}[1]{\textcolor[rgb]{0.25,0.44,0.63}{{#1}}}
    \newcommand{\CommentTok}[1]{\textcolor[rgb]{0.38,0.63,0.69}{\textit{{#1}}}}
    \newcommand{\OtherTok}[1]{\textcolor[rgb]{0.00,0.44,0.13}{{#1}}}
    \newcommand{\AlertTok}[1]{\textcolor[rgb]{1.00,0.00,0.00}{\textbf{{#1}}}}
    \newcommand{\FunctionTok}[1]{\textcolor[rgb]{0.02,0.16,0.49}{{#1}}}
    \newcommand{\RegionMarkerTok}[1]{{#1}}
    \newcommand{\ErrorTok}[1]{\textcolor[rgb]{1.00,0.00,0.00}{\textbf{{#1}}}}
    \newcommand{\NormalTok}[1]{{#1}}

    % Additional commands for more recent versions of Pandoc
    \newcommand{\ConstantTok}[1]{\textcolor[rgb]{0.53,0.00,0.00}{{#1}}}
    \newcommand{\SpecialCharTok}[1]{\textcolor[rgb]{0.25,0.44,0.63}{{#1}}}
    \newcommand{\VerbatimStringTok}[1]{\textcolor[rgb]{0.25,0.44,0.63}{{#1}}}
    \newcommand{\SpecialStringTok}[1]{\textcolor[rgb]{0.73,0.40,0.53}{{#1}}}
    \newcommand{\ImportTok}[1]{{#1}}
    \newcommand{\DocumentationTok}[1]{\textcolor[rgb]{0.73,0.13,0.13}{\textit{{#1}}}}
    \newcommand{\AnnotationTok}[1]{\textcolor[rgb]{0.38,0.63,0.69}{\textbf{\textit{{#1}}}}}
    \newcommand{\CommentVarTok}[1]{\textcolor[rgb]{0.38,0.63,0.69}{\textbf{\textit{{#1}}}}}
    \newcommand{\VariableTok}[1]{\textcolor[rgb]{0.10,0.09,0.49}{{#1}}}
    \newcommand{\ControlFlowTok}[1]{\textcolor[rgb]{0.00,0.44,0.13}{\textbf{{#1}}}}
    \newcommand{\OperatorTok}[1]{\textcolor[rgb]{0.40,0.40,0.40}{{#1}}}
    \newcommand{\BuiltInTok}[1]{{#1}}
    \newcommand{\ExtensionTok}[1]{{#1}}
    \newcommand{\PreprocessorTok}[1]{\textcolor[rgb]{0.74,0.48,0.00}{{#1}}}
    \newcommand{\AttributeTok}[1]{\textcolor[rgb]{0.49,0.56,0.16}{{#1}}}
    \newcommand{\InformationTok}[1]{\textcolor[rgb]{0.38,0.63,0.69}{\textbf{\textit{{#1}}}}}
    \newcommand{\WarningTok}[1]{\textcolor[rgb]{0.38,0.63,0.69}{\textbf{\textit{{#1}}}}}


    % Define a nice break command that doesn't care if a line doesn't already
    % exist.
    \def\br{\hspace*{\fill} \\* }
    % Math Jax compatibility definitions
    \def\gt{>}
    \def\lt{<}
    \let\Oldtex\TeX
    \let\Oldlatex\LaTeX
    \renewcommand{\TeX}{\textrm{\Oldtex}}
    \renewcommand{\LaTeX}{\textrm{\Oldlatex}}
    % Document parameters
    % Document title
    \title{algorytmy\_rojowe}
    
    
    
    
    
    
    
% Pygments definitions
\makeatletter
\def\PY@reset{\let\PY@it=\relax \let\PY@bf=\relax%
    \let\PY@ul=\relax \let\PY@tc=\relax%
    \let\PY@bc=\relax \let\PY@ff=\relax}
\def\PY@tok#1{\csname PY@tok@#1\endcsname}
\def\PY@toks#1+{\ifx\relax#1\empty\else%
    \PY@tok{#1}\expandafter\PY@toks\fi}
\def\PY@do#1{\PY@bc{\PY@tc{\PY@ul{%
    \PY@it{\PY@bf{\PY@ff{#1}}}}}}}
\def\PY#1#2{\PY@reset\PY@toks#1+\relax+\PY@do{#2}}

\@namedef{PY@tok@w}{\def\PY@tc##1{\textcolor[rgb]{0.73,0.73,0.73}{##1}}}
\@namedef{PY@tok@c}{\let\PY@it=\textit\def\PY@tc##1{\textcolor[rgb]{0.24,0.48,0.48}{##1}}}
\@namedef{PY@tok@cp}{\def\PY@tc##1{\textcolor[rgb]{0.61,0.40,0.00}{##1}}}
\@namedef{PY@tok@k}{\let\PY@bf=\textbf\def\PY@tc##1{\textcolor[rgb]{0.00,0.50,0.00}{##1}}}
\@namedef{PY@tok@kp}{\def\PY@tc##1{\textcolor[rgb]{0.00,0.50,0.00}{##1}}}
\@namedef{PY@tok@kt}{\def\PY@tc##1{\textcolor[rgb]{0.69,0.00,0.25}{##1}}}
\@namedef{PY@tok@o}{\def\PY@tc##1{\textcolor[rgb]{0.40,0.40,0.40}{##1}}}
\@namedef{PY@tok@ow}{\let\PY@bf=\textbf\def\PY@tc##1{\textcolor[rgb]{0.67,0.13,1.00}{##1}}}
\@namedef{PY@tok@nb}{\def\PY@tc##1{\textcolor[rgb]{0.00,0.50,0.00}{##1}}}
\@namedef{PY@tok@nf}{\def\PY@tc##1{\textcolor[rgb]{0.00,0.00,1.00}{##1}}}
\@namedef{PY@tok@nc}{\let\PY@bf=\textbf\def\PY@tc##1{\textcolor[rgb]{0.00,0.00,1.00}{##1}}}
\@namedef{PY@tok@nn}{\let\PY@bf=\textbf\def\PY@tc##1{\textcolor[rgb]{0.00,0.00,1.00}{##1}}}
\@namedef{PY@tok@ne}{\let\PY@bf=\textbf\def\PY@tc##1{\textcolor[rgb]{0.80,0.25,0.22}{##1}}}
\@namedef{PY@tok@nv}{\def\PY@tc##1{\textcolor[rgb]{0.10,0.09,0.49}{##1}}}
\@namedef{PY@tok@no}{\def\PY@tc##1{\textcolor[rgb]{0.53,0.00,0.00}{##1}}}
\@namedef{PY@tok@nl}{\def\PY@tc##1{\textcolor[rgb]{0.46,0.46,0.00}{##1}}}
\@namedef{PY@tok@ni}{\let\PY@bf=\textbf\def\PY@tc##1{\textcolor[rgb]{0.44,0.44,0.44}{##1}}}
\@namedef{PY@tok@na}{\def\PY@tc##1{\textcolor[rgb]{0.41,0.47,0.13}{##1}}}
\@namedef{PY@tok@nt}{\let\PY@bf=\textbf\def\PY@tc##1{\textcolor[rgb]{0.00,0.50,0.00}{##1}}}
\@namedef{PY@tok@nd}{\def\PY@tc##1{\textcolor[rgb]{0.67,0.13,1.00}{##1}}}
\@namedef{PY@tok@s}{\def\PY@tc##1{\textcolor[rgb]{0.73,0.13,0.13}{##1}}}
\@namedef{PY@tok@sd}{\let\PY@it=\textit\def\PY@tc##1{\textcolor[rgb]{0.73,0.13,0.13}{##1}}}
\@namedef{PY@tok@si}{\let\PY@bf=\textbf\def\PY@tc##1{\textcolor[rgb]{0.64,0.35,0.47}{##1}}}
\@namedef{PY@tok@se}{\let\PY@bf=\textbf\def\PY@tc##1{\textcolor[rgb]{0.67,0.36,0.12}{##1}}}
\@namedef{PY@tok@sr}{\def\PY@tc##1{\textcolor[rgb]{0.64,0.35,0.47}{##1}}}
\@namedef{PY@tok@ss}{\def\PY@tc##1{\textcolor[rgb]{0.10,0.09,0.49}{##1}}}
\@namedef{PY@tok@sx}{\def\PY@tc##1{\textcolor[rgb]{0.00,0.50,0.00}{##1}}}
\@namedef{PY@tok@m}{\def\PY@tc##1{\textcolor[rgb]{0.40,0.40,0.40}{##1}}}
\@namedef{PY@tok@gh}{\let\PY@bf=\textbf\def\PY@tc##1{\textcolor[rgb]{0.00,0.00,0.50}{##1}}}
\@namedef{PY@tok@gu}{\let\PY@bf=\textbf\def\PY@tc##1{\textcolor[rgb]{0.50,0.00,0.50}{##1}}}
\@namedef{PY@tok@gd}{\def\PY@tc##1{\textcolor[rgb]{0.63,0.00,0.00}{##1}}}
\@namedef{PY@tok@gi}{\def\PY@tc##1{\textcolor[rgb]{0.00,0.52,0.00}{##1}}}
\@namedef{PY@tok@gr}{\def\PY@tc##1{\textcolor[rgb]{0.89,0.00,0.00}{##1}}}
\@namedef{PY@tok@ge}{\let\PY@it=\textit}
\@namedef{PY@tok@gs}{\let\PY@bf=\textbf}
\@namedef{PY@tok@gp}{\let\PY@bf=\textbf\def\PY@tc##1{\textcolor[rgb]{0.00,0.00,0.50}{##1}}}
\@namedef{PY@tok@go}{\def\PY@tc##1{\textcolor[rgb]{0.44,0.44,0.44}{##1}}}
\@namedef{PY@tok@gt}{\def\PY@tc##1{\textcolor[rgb]{0.00,0.27,0.87}{##1}}}
\@namedef{PY@tok@err}{\def\PY@bc##1{{\setlength{\fboxsep}{\string -\fboxrule}\fcolorbox[rgb]{1.00,0.00,0.00}{1,1,1}{\strut ##1}}}}
\@namedef{PY@tok@kc}{\let\PY@bf=\textbf\def\PY@tc##1{\textcolor[rgb]{0.00,0.50,0.00}{##1}}}
\@namedef{PY@tok@kd}{\let\PY@bf=\textbf\def\PY@tc##1{\textcolor[rgb]{0.00,0.50,0.00}{##1}}}
\@namedef{PY@tok@kn}{\let\PY@bf=\textbf\def\PY@tc##1{\textcolor[rgb]{0.00,0.50,0.00}{##1}}}
\@namedef{PY@tok@kr}{\let\PY@bf=\textbf\def\PY@tc##1{\textcolor[rgb]{0.00,0.50,0.00}{##1}}}
\@namedef{PY@tok@bp}{\def\PY@tc##1{\textcolor[rgb]{0.00,0.50,0.00}{##1}}}
\@namedef{PY@tok@fm}{\def\PY@tc##1{\textcolor[rgb]{0.00,0.00,1.00}{##1}}}
\@namedef{PY@tok@vc}{\def\PY@tc##1{\textcolor[rgb]{0.10,0.09,0.49}{##1}}}
\@namedef{PY@tok@vg}{\def\PY@tc##1{\textcolor[rgb]{0.10,0.09,0.49}{##1}}}
\@namedef{PY@tok@vi}{\def\PY@tc##1{\textcolor[rgb]{0.10,0.09,0.49}{##1}}}
\@namedef{PY@tok@vm}{\def\PY@tc##1{\textcolor[rgb]{0.10,0.09,0.49}{##1}}}
\@namedef{PY@tok@sa}{\def\PY@tc##1{\textcolor[rgb]{0.73,0.13,0.13}{##1}}}
\@namedef{PY@tok@sb}{\def\PY@tc##1{\textcolor[rgb]{0.73,0.13,0.13}{##1}}}
\@namedef{PY@tok@sc}{\def\PY@tc##1{\textcolor[rgb]{0.73,0.13,0.13}{##1}}}
\@namedef{PY@tok@dl}{\def\PY@tc##1{\textcolor[rgb]{0.73,0.13,0.13}{##1}}}
\@namedef{PY@tok@s2}{\def\PY@tc##1{\textcolor[rgb]{0.73,0.13,0.13}{##1}}}
\@namedef{PY@tok@sh}{\def\PY@tc##1{\textcolor[rgb]{0.73,0.13,0.13}{##1}}}
\@namedef{PY@tok@s1}{\def\PY@tc##1{\textcolor[rgb]{0.73,0.13,0.13}{##1}}}
\@namedef{PY@tok@mb}{\def\PY@tc##1{\textcolor[rgb]{0.40,0.40,0.40}{##1}}}
\@namedef{PY@tok@mf}{\def\PY@tc##1{\textcolor[rgb]{0.40,0.40,0.40}{##1}}}
\@namedef{PY@tok@mh}{\def\PY@tc##1{\textcolor[rgb]{0.40,0.40,0.40}{##1}}}
\@namedef{PY@tok@mi}{\def\PY@tc##1{\textcolor[rgb]{0.40,0.40,0.40}{##1}}}
\@namedef{PY@tok@il}{\def\PY@tc##1{\textcolor[rgb]{0.40,0.40,0.40}{##1}}}
\@namedef{PY@tok@mo}{\def\PY@tc##1{\textcolor[rgb]{0.40,0.40,0.40}{##1}}}
\@namedef{PY@tok@ch}{\let\PY@it=\textit\def\PY@tc##1{\textcolor[rgb]{0.24,0.48,0.48}{##1}}}
\@namedef{PY@tok@cm}{\let\PY@it=\textit\def\PY@tc##1{\textcolor[rgb]{0.24,0.48,0.48}{##1}}}
\@namedef{PY@tok@cpf}{\let\PY@it=\textit\def\PY@tc##1{\textcolor[rgb]{0.24,0.48,0.48}{##1}}}
\@namedef{PY@tok@c1}{\let\PY@it=\textit\def\PY@tc##1{\textcolor[rgb]{0.24,0.48,0.48}{##1}}}
\@namedef{PY@tok@cs}{\let\PY@it=\textit\def\PY@tc##1{\textcolor[rgb]{0.24,0.48,0.48}{##1}}}

\def\PYZbs{\char`\\}
\def\PYZus{\char`\_}
\def\PYZob{\char`\{}
\def\PYZcb{\char`\}}
\def\PYZca{\char`\^}
\def\PYZam{\char`\&}
\def\PYZlt{\char`\<}
\def\PYZgt{\char`\>}
\def\PYZsh{\char`\#}
\def\PYZpc{\char`\%}
\def\PYZdl{\char`\$}
\def\PYZhy{\char`\-}
\def\PYZsq{\char`\'}
\def\PYZdq{\char`\"}
\def\PYZti{\char`\~}
% for compatibility with earlier versions
\def\PYZat{@}
\def\PYZlb{[}
\def\PYZrb{]}
\makeatother


    % For linebreaks inside Verbatim environment from package fancyvrb.
    \makeatletter
        \newbox\Wrappedcontinuationbox
        \newbox\Wrappedvisiblespacebox
        \newcommand*\Wrappedvisiblespace {\textcolor{red}{\textvisiblespace}}
        \newcommand*\Wrappedcontinuationsymbol {\textcolor{red}{\llap{\tiny$\m@th\hookrightarrow$}}}
        \newcommand*\Wrappedcontinuationindent {3ex }
        \newcommand*\Wrappedafterbreak {\kern\Wrappedcontinuationindent\copy\Wrappedcontinuationbox}
        % Take advantage of the already applied Pygments mark-up to insert
        % potential linebreaks for TeX processing.
        %        {, <, #, %, $, ' and ": go to next line.
        %        _, }, ^, &, >, - and ~: stay at end of broken line.
        % Use of \textquotesingle for straight quote.
        \newcommand*\Wrappedbreaksatspecials {%
            \def\PYGZus{\discretionary{\char`\_}{\Wrappedafterbreak}{\char`\_}}%
            \def\PYGZob{\discretionary{}{\Wrappedafterbreak\char`\{}{\char`\{}}%
            \def\PYGZcb{\discretionary{\char`\}}{\Wrappedafterbreak}{\char`\}}}%
            \def\PYGZca{\discretionary{\char`\^}{\Wrappedafterbreak}{\char`\^}}%
            \def\PYGZam{\discretionary{\char`\&}{\Wrappedafterbreak}{\char`\&}}%
            \def\PYGZlt{\discretionary{}{\Wrappedafterbreak\char`\<}{\char`\<}}%
            \def\PYGZgt{\discretionary{\char`\>}{\Wrappedafterbreak}{\char`\>}}%
            \def\PYGZsh{\discretionary{}{\Wrappedafterbreak\char`\#}{\char`\#}}%
            \def\PYGZpc{\discretionary{}{\Wrappedafterbreak\char`\%}{\char`\%}}%
            \def\PYGZdl{\discretionary{}{\Wrappedafterbreak\char`\$}{\char`\$}}%
            \def\PYGZhy{\discretionary{\char`\-}{\Wrappedafterbreak}{\char`\-}}%
            \def\PYGZsq{\discretionary{}{\Wrappedafterbreak\textquotesingle}{\textquotesingle}}%
            \def\PYGZdq{\discretionary{}{\Wrappedafterbreak\char`\"}{\char`\"}}%
            \def\PYGZti{\discretionary{\char`\~}{\Wrappedafterbreak}{\char`\~}}%
        }
        % Some characters . , ; ? ! / are not pygmentized.
        % This macro makes them "active" and they will insert potential linebreaks
        \newcommand*\Wrappedbreaksatpunct {%
            \lccode`\~`\.\lowercase{\def~}{\discretionary{\hbox{\char`\.}}{\Wrappedafterbreak}{\hbox{\char`\.}}}%
            \lccode`\~`\,\lowercase{\def~}{\discretionary{\hbox{\char`\,}}{\Wrappedafterbreak}{\hbox{\char`\,}}}%
            \lccode`\~`\;\lowercase{\def~}{\discretionary{\hbox{\char`\;}}{\Wrappedafterbreak}{\hbox{\char`\;}}}%
            \lccode`\~`\:\lowercase{\def~}{\discretionary{\hbox{\char`\:}}{\Wrappedafterbreak}{\hbox{\char`\:}}}%
            \lccode`\~`\?\lowercase{\def~}{\discretionary{\hbox{\char`\?}}{\Wrappedafterbreak}{\hbox{\char`\?}}}%
            \lccode`\~`\!\lowercase{\def~}{\discretionary{\hbox{\char`\!}}{\Wrappedafterbreak}{\hbox{\char`\!}}}%
            \lccode`\~`\/\lowercase{\def~}{\discretionary{\hbox{\char`\/}}{\Wrappedafterbreak}{\hbox{\char`\/}}}%
            \catcode`\.\active
            \catcode`\,\active
            \catcode`\;\active
            \catcode`\:\active
            \catcode`\?\active
            \catcode`\!\active
            \catcode`\/\active
            \lccode`\~`\~
        }
    \makeatother

    \let\OriginalVerbatim=\Verbatim
    \makeatletter
    \renewcommand{\Verbatim}[1][1]{%
        %\parskip\z@skip
        \sbox\Wrappedcontinuationbox {\Wrappedcontinuationsymbol}%
        \sbox\Wrappedvisiblespacebox {\FV@SetupFont\Wrappedvisiblespace}%
        \def\FancyVerbFormatLine ##1{\hsize\linewidth
            \vtop{\raggedright\hyphenpenalty\z@\exhyphenpenalty\z@
                \doublehyphendemerits\z@\finalhyphendemerits\z@
                \strut ##1\strut}%
        }%
        % If the linebreak is at a space, the latter will be displayed as visible
        % space at end of first line, and a continuation symbol starts next line.
        % Stretch/shrink are however usually zero for typewriter font.
        \def\FV@Space {%
            \nobreak\hskip\z@ plus\fontdimen3\font minus\fontdimen4\font
            \discretionary{\copy\Wrappedvisiblespacebox}{\Wrappedafterbreak}
            {\kern\fontdimen2\font}%
        }%

        % Allow breaks at special characters using \PYG... macros.
        \Wrappedbreaksatspecials
        % Breaks at punctuation characters . , ; ? ! and / need catcode=\active
        \OriginalVerbatim[#1,codes*=\Wrappedbreaksatpunct]%
    }
    \makeatother

    % Exact colors from NB
    \definecolor{incolor}{HTML}{303F9F}
    \definecolor{outcolor}{HTML}{D84315}
    \definecolor{cellborder}{HTML}{CFCFCF}
    \definecolor{cellbackground}{HTML}{F7F7F7}

    % prompt
    \makeatletter
    \newcommand{\boxspacing}{\kern\kvtcb@left@rule\kern\kvtcb@boxsep}
    \makeatother
    \newcommand{\prompt}[4]{
        {\ttfamily\llap{{\color{#2}[#3]:\hspace{3pt}#4}}\vspace{-\baselineskip}}
    }
    

    
    % Prevent overflowing lines due to hard-to-break entities
    \sloppy
    % Setup hyperref package
    \hypersetup{
      breaklinks=true,  % so long urls are correctly broken across lines
      colorlinks=true,
      urlcolor=urlcolor,
      linkcolor=linkcolor,
      citecolor=citecolor,
      }
    % Slightly bigger margins than the latex defaults
    
    \geometry{verbose,tmargin=1in,bmargin=1in,lmargin=1in,rmargin=1in}
    
    

\begin{document}
    
    \maketitle
    
    

    
    \#Algorytmy rojowe

    \hypertarget{link-do-repozytorium-na-github}{%
\subsection{Link do repozytorium na
github}\label{link-do-repozytorium-na-github}}

\begin{itemize}
\tightlist
\item
  Algorytm PSO
  https://github.com/MaciejSzerwinski/PSO\_Algorithm\_AI.git
\item
  Algorytm mrówkowy https://github.com/kingasmi/algorytm\_mrowkowy
\end{itemize}

    \hypertarget{algorytm-pso}{%
\section{Algorytm PSO}\label{algorytm-pso}}

    \hypertarget{teoria}{%
\subsection{1. Teoria}\label{teoria}}

\hypertarget{wstux119p}{%
\subsubsection{1.1 Wstęp}\label{wstux119p}}

Optymalizacja za pomocą roju cząstek (ang. Particle Swarm Optimization,
w skrócie PSO) to algorytm metaheurystyczny służący do rozwiązywania
problemów optymalizacyjnych.

Problem optymalizacyjny to problem, którego rozwiązanie polega na
odnalezieniu optymalnej (największej lub najmniejszej) wartości pewnej
funkcji, zwanej funkcją celu. Zakres wartości argumentów tej funkcji
nazywany jest przestrzenią rozwiązań. Pojedynczy punkt w tej
przestrzeni, wyznaczony przez ustalone wartości poszczególnych
argumentów nazywany jest rozwiązaniem.

Przykładem problemu optymalizacyjnego jest problem plecakowy. Mając
plecak o określonej pojemności oraz zestaw przedmiotów posiadających
określoną wartość i rozmiar, należy określić zbiór przedmiotów o
największej możliwej wartości, bez przekraczania pojemności plecaka. W
podanym przykładzie rozwiązaniem jest jeden określony podzbiór
przedmiotów, natomiast funkcje celu określa ich łączna wartość.
Przestrzeń rozwiązań stanowi zbiór wszystkich możliwych kombinacji
przedmiotów mieszczących się w plecaku.

Algorytmy metaheurystyczne, albo krócej metaheurystyki to algorytmy
``uniwersalne'', pozwalające na rozwiązywanie dowolnych problemów
obliczeniowych. Metaheurystyki nie gwarantują odnalezienia optymalnego
rozwiązania, a jedynie rozwiązania zbliżonego do optymalnego.
Wykorzystywane są w sytuacjach, gdy uzyskanie najlepszego rozwiązania
byłoby zbyt kosztowne obliczeniowo.

\hypertarget{zasady-dziaux142ania-algorytmu-pso}{%
\subsubsection{1.2 Zasady działania algorytmu
PSO}\label{zasady-dziaux142ania-algorytmu-pso}}

Ideą algorytmu PSO jest iteracyjne przeszukiwanie przestrzeni rozwiązań
problemu przy pomocy roju cząstek. Każda z cząstek posiada swoją pozycję
w przestrzeni rozwiązań, prędkość oraz kierunek w jakim się porusza.
Ponadto zapamiętywane jest najlepsze rozwiązanie znalezione do tej pory
przez każdą z cząstek (rozwiązanie lokalne), a także najlepsze
rozwiązanie z całego roju (rozwiązanie globalne). Prędkość ruchu
poszczególnych cząstek zależy od położenia najlepszego globalnego i
lokalnego rozwiązania oraz od prędkości w poprzednich krokach. Poniżej
przedstawiony jest wzór pozwalający na obliczenie prędkości danej
cząstki.

\[
v \leftarrow \omega v + \phi l_rl(l-x) + \phi gr_g(g-x)
\]

Gdzie: * v - prędkość cząstki * ω - współczynnik bezwładności, określa
wpływ prędkości w poprzednim kroku * φl - współczynnik dążenia do
najlepszego lokalnego rozwiązania * φg - współczynnik dążenia do
najlepszego globalnego rozwiązania * l - położenie najlepszego lokalnego
rozwiązania * g - położenie najlepszego globalnego rozwiązania * x -
położenie cząstki * rl, rg - losowe wartości z przedziału
\textless0,1\textgreater{} Powyższy wzór pozwala na aktualizacje
prędkości wszystkich cząstek na podstawie uzyskanej do tej pory wiedzy.

\hypertarget{schemat-dziaux142ania-algorytmu-pso}{%
\subsubsection{1.3 Schemat działania algorytmu
PSO}\label{schemat-dziaux142ania-algorytmu-pso}}

Schemat działania algorytmu przedstawia się następująco:

\begin{itemize}
\tightlist
\item
  Dla każdej cząstki ze zbioru:

  \begin{itemize}
  \tightlist
  \item
    Wylosuj pozycje początkową z przestrzeni rozwiązań
  \item
    Zapisz aktualną pozycje cząstki jako najlepsze lokalne rozwiązanie
  \item
    Jeśli rozwiązanie to jest lepsze od najlepszego rozwiązanie
    globalnego, to zapisz je jako najlepsze
  \item
    Wylosuj prędkość początkową
  \end{itemize}
\item
  Dopóki nie zostanie spełniony warunek stopu (np. minie określona
  liczba iteracji):

  \begin{itemize}
  \tightlist
  \item
    Dla każdej cząstki ze zbioru:

    \begin{itemize}
    \tightlist
    \item
      Wybierz losowe wartości parametrów rl i rg
    \item
      Zaktualizuj prędkość cząstki wg powyższego wzoru
    \item
      Zaktualizuj położenie cząstki w przestrzeni
    \item
      Jeśli aktualne rozwiązanie jest lepsze od najlepszego rozwiązania
      lokalnego:

      \begin{itemize}
      \tightlist
      \item
        Zapisz aktualne rozwiązanie jako najlepsze lokalnie
      \end{itemize}
    \item
      Jeśli aktualne rozwiązanie jest lepsze od najlepszego rozwiązania
      globalnego:

      \begin{itemize}
      \tightlist
      \item
        Zapisz aktualne rozwiązanie jako najlepsze globalnie
      \end{itemize}
    \end{itemize}
  \end{itemize}
\end{itemize}

    \hypertarget{rozwiux105zanie}{%
\subsection{2. Rozwiązanie}\label{rozwiux105zanie}}

    \hypertarget{importowanie-potrzebnych-bibliotek}{%
\subsection{2.1 Importowanie potrzebnych
bibliotek}\label{importowanie-potrzebnych-bibliotek}}

    \begin{tcolorbox}[breakable, size=fbox, boxrule=1pt, pad at break*=1mm,colback=cellbackground, colframe=cellborder]
\prompt{In}{incolor}{ }{\boxspacing}
\begin{Verbatim}[commandchars=\\\{\}]
\PY{k+kn}{import} \PY{n+nn}{numpy} \PY{k}{as} \PY{n+nn}{np}
\PY{k+kn}{import} \PY{n+nn}{matplotlib}\PY{n+nn}{.}\PY{n+nn}{pyplot} \PY{k}{as} \PY{n+nn}{plt}
\PY{k+kn}{from} \PY{n+nn}{matplotlib}\PY{n+nn}{.}\PY{n+nn}{animation} \PY{k+kn}{import} \PY{n}{FuncAnimation}
\PY{k+kn}{from} \PY{n+nn}{mpl\PYZus{}toolkits}\PY{n+nn}{.}\PY{n+nn}{mplot3d} \PY{k+kn}{import} \PY{n}{Axes3D}
\end{Verbatim}
\end{tcolorbox}

    \hypertarget{zdefiniowanie-funkcji-celu}{%
\subsection{2.2 Zdefiniowanie funkcji
celu}\label{zdefiniowanie-funkcji-celu}}

Funkcja ``sphere\_function'' oblicza wartość funkcji celu dla podanych
wartości x i y. Funkcja ta jest reprezentacją funkcji sfery, która jest
zdefiniowana jako suma kwadratów wartości x i y. Zwraca wynik tej sumy.

Funkcja ``rastrigin\_function'' również oblicza wartość funkcji celu dla
podanych wartości x i y. Ta funkcja reprezentuje funkcję Rastrigina,
która jest zdefiniowana jako suma kilku składników. Pierwszy składnik to
20, a pozostałe składniki zawierają kwadraty wartości x i y, a także
obliczenia kosinusów na podstawie tych wartości. Funkcja zwraca wynik
sumy tych składników.

    \begin{tcolorbox}[breakable, size=fbox, boxrule=1pt, pad at break*=1mm,colback=cellbackground, colframe=cellborder]
\prompt{In}{incolor}{ }{\boxspacing}
\begin{Verbatim}[commandchars=\\\{\}]
\PY{k}{def} \PY{n+nf}{sphere\PYZus{}function}\PY{p}{(}\PY{n}{x}\PY{p}{,} \PY{n}{y}\PY{p}{)}\PY{p}{:}
    \PY{l+s+sd}{\PYZdq{}\PYZdq{}\PYZdq{}Funkcja celu \PYZhy{} sfera\PYZdq{}\PYZdq{}\PYZdq{}}
    \PY{k}{return} \PY{n}{x}\PY{o}{*}\PY{o}{*}\PY{l+m+mi}{2} \PY{o}{+} \PY{n}{y}\PY{o}{*}\PY{o}{*}\PY{l+m+mi}{2}

\PY{k}{def} \PY{n+nf}{rastrigin\PYZus{}function}\PY{p}{(}\PY{n}{x}\PY{p}{,} \PY{n}{y}\PY{p}{)}\PY{p}{:}
    \PY{l+s+sd}{\PYZdq{}\PYZdq{}\PYZdq{}Funkcja celu \PYZhy{} Rastrigin\PYZdq{}\PYZdq{}\PYZdq{}}
    \PY{k}{return} \PY{l+m+mi}{20} \PY{o}{+} \PY{n}{x}\PY{o}{*}\PY{o}{*}\PY{l+m+mi}{2} \PY{o}{\PYZhy{}} \PY{l+m+mi}{10} \PY{o}{*} \PY{n}{np}\PY{o}{.}\PY{n}{cos}\PY{p}{(}\PY{l+m+mi}{2} \PY{o}{*} \PY{n}{np}\PY{o}{.}\PY{n}{pi} \PY{o}{*} \PY{n}{x}\PY{p}{)} \PY{o}{+} \PY{n}{y}\PY{o}{*}\PY{o}{*}\PY{l+m+mi}{2} \PY{o}{\PYZhy{}} \PY{l+m+mi}{10} \PY{o}{*} \PY{n}{np}\PY{o}{.}\PY{n}{cos}\PY{p}{(}\PY{l+m+mi}{2} \PY{o}{*} \PY{n}{np}\PY{o}{.}\PY{n}{pi} \PY{o}{*} \PY{n}{y}\PY{p}{)}
\end{Verbatim}
\end{tcolorbox}

    \hypertarget{zaimplementowanie-klasy-odpowiadajux105cej-za-tworzenie-czux105steczek-roju}{%
\subsection{2.3 Zaimplementowanie klasy odpowiadającej za tworzenie
cząsteczek
roju}\label{zaimplementowanie-klasy-odpowiadajux105cej-za-tworzenie-czux105steczek-roju}}

Metoda init to konstruktor klasy Particle, która inicjalizuje atrybuty
cząstki.

\begin{itemize}
\item
  Atrybut self.position jest inicjalizowany jako losowe wartości z
  przedziału {[}-10, 10{]} dla obu wymiarów.
\item
  Atrybut self.velocity jest inicjalizowany jako losowe wartości z
  przedziału {[}-1, 1{]} dla obu wymiarów.
\item
  Atrybut self.best\_position jest inicjalizowany jako początkowe
  położenie cząstki.
\item
  Atrybut self.best\_score jest inicjalizowany jako wartość funkcji celu
  dla początkowego położenia cząstki. Metoda update\_velocity
  aktualizuje prędkość cząstki na podstawie najlepszego położenia
  cząstki (self.best\_position), najlepszego położenia globalnego
  (global\_best\_position) oraz wag: inertia\_weight, cognitive\_weight
  i social\_weight.
\item
  R1 i R2 są losowymi wektorami o długości 2.
\item
  cognitive\_component to składnik kognitywny, który jest iloczynem wag
  cognitive\_weight, losowego wektora r1 oraz różnicy między najlepszym
  położeniem cząstki a jej aktualnym położeniem (self.best\_position -
  self.position).
\item
  social\_component to składnik społeczny, który jest iloczynem wag
  social\_weight, losowego wektora r2 oraz różnicy między najlepszym
  położeniem globalnym a aktualnym położeniem cząstki
  (global\_best\_position - self.position).
\item
  Nowa prędkość cząstki jest obliczana jako suma prędkości inercyjnej
  (inertia\_weight * self.velocity) oraz składników kognitywnego i
  społecznego.
\end{itemize}

Metoda update\_position aktualizuje położenie cząstki na podstawie jej
prędkości.

\begin{itemize}
\tightlist
\item
  Nowe położenie cząstki jest obliczane poprzez dodanie prędkości do
  aktualnego położenia (self.position += self.velocity).
\item
  Następnie położenie jest ograniczane do przedziału {[}-10, 10{]} przy
  użyciu funkcji np.clip.
\item
  Obliczany jest nowy wynik funkcji celu dla nowego położenia cząstki
  (current\_score = sphere\_function(*self.position)).
\item
  Jeśli nowy wynik jest lepszy niż dotychczasowy najlepszy wynik cząstki
  (current\_score \textless{} self.best\_score), to aktualizowane są
  atrybuty self.best\_position i self.best\_score.
\end{itemize}

    \begin{tcolorbox}[breakable, size=fbox, boxrule=1pt, pad at break*=1mm,colback=cellbackground, colframe=cellborder]
\prompt{In}{incolor}{ }{\boxspacing}
\begin{Verbatim}[commandchars=\\\{\}]
\PY{k}{class} \PY{n+nc}{Particle}\PY{p}{:}
    \PY{k}{def} \PY{n+nf+fm}{\PYZus{}\PYZus{}init\PYZus{}\PYZus{}}\PY{p}{(}\PY{n+nb+bp}{self}\PY{p}{)}\PY{p}{:}
        \PY{n+nb+bp}{self}\PY{o}{.}\PY{n}{position} \PY{o}{=} \PY{n}{np}\PY{o}{.}\PY{n}{random}\PY{o}{.}\PY{n}{uniform}\PY{p}{(}\PY{o}{\PYZhy{}}\PY{l+m+mi}{10}\PY{p}{,} \PY{l+m+mi}{10}\PY{p}{,} \PY{l+m+mi}{2}\PY{p}{)}
        \PY{n+nb+bp}{self}\PY{o}{.}\PY{n}{velocity} \PY{o}{=} \PY{n}{np}\PY{o}{.}\PY{n}{random}\PY{o}{.}\PY{n}{uniform}\PY{p}{(}\PY{o}{\PYZhy{}}\PY{l+m+mi}{1}\PY{p}{,} \PY{l+m+mi}{1}\PY{p}{,} \PY{l+m+mi}{2}\PY{p}{)}
        \PY{n+nb+bp}{self}\PY{o}{.}\PY{n}{best\PYZus{}position} \PY{o}{=} \PY{n+nb+bp}{self}\PY{o}{.}\PY{n}{position}
        \PY{n+nb+bp}{self}\PY{o}{.}\PY{n}{best\PYZus{}score} \PY{o}{=} \PY{n}{sphere\PYZus{}function}\PY{p}{(}\PY{o}{*}\PY{n+nb+bp}{self}\PY{o}{.}\PY{n}{position}\PY{p}{)}

    \PY{k}{def} \PY{n+nf}{update\PYZus{}velocity}\PY{p}{(}\PY{n+nb+bp}{self}\PY{p}{,} \PY{n}{global\PYZus{}best\PYZus{}position}\PY{p}{,} \PY{n}{inertia\PYZus{}weight}\PY{p}{,} \PY{n}{cognitive\PYZus{}weight}\PY{p}{,} \PY{n}{social\PYZus{}weight}\PY{p}{)}\PY{p}{:}
        \PY{n}{r1} \PY{o}{=} \PY{n}{np}\PY{o}{.}\PY{n}{random}\PY{o}{.}\PY{n}{rand}\PY{p}{(}\PY{l+m+mi}{2}\PY{p}{)}
        \PY{n}{r2} \PY{o}{=} \PY{n}{np}\PY{o}{.}\PY{n}{random}\PY{o}{.}\PY{n}{rand}\PY{p}{(}\PY{l+m+mi}{2}\PY{p}{)}

        \PY{n}{cognitive\PYZus{}component} \PY{o}{=} \PY{n}{cognitive\PYZus{}weight} \PY{o}{*} \PY{n}{r1} \PY{o}{*} \PY{p}{(}\PY{n+nb+bp}{self}\PY{o}{.}\PY{n}{best\PYZus{}position} \PY{o}{\PYZhy{}} \PY{n+nb+bp}{self}\PY{o}{.}\PY{n}{position}\PY{p}{)}
        \PY{n}{social\PYZus{}component} \PY{o}{=} \PY{n}{social\PYZus{}weight} \PY{o}{*} \PY{n}{r2} \PY{o}{*} \PY{p}{(}\PY{n}{global\PYZus{}best\PYZus{}position} \PY{o}{\PYZhy{}} \PY{n+nb+bp}{self}\PY{o}{.}\PY{n}{position}\PY{p}{)}
        \PY{n+nb+bp}{self}\PY{o}{.}\PY{n}{velocity} \PY{o}{=} \PY{n}{inertia\PYZus{}weight} \PY{o}{*} \PY{n+nb+bp}{self}\PY{o}{.}\PY{n}{velocity} \PY{o}{+} \PY{n}{cognitive\PYZus{}component} \PY{o}{+} \PY{n}{social\PYZus{}component}

    \PY{k}{def} \PY{n+nf}{update\PYZus{}position}\PY{p}{(}\PY{n+nb+bp}{self}\PY{p}{)}\PY{p}{:}
        \PY{n+nb+bp}{self}\PY{o}{.}\PY{n}{position} \PY{o}{+}\PY{o}{=} \PY{n+nb+bp}{self}\PY{o}{.}\PY{n}{velocity}
        \PY{n+nb+bp}{self}\PY{o}{.}\PY{n}{position} \PY{o}{=} \PY{n}{np}\PY{o}{.}\PY{n}{clip}\PY{p}{(}\PY{n+nb+bp}{self}\PY{o}{.}\PY{n}{position}\PY{p}{,} \PY{o}{\PYZhy{}}\PY{l+m+mi}{10}\PY{p}{,} \PY{l+m+mi}{10}\PY{p}{)}
        \PY{n}{current\PYZus{}score} \PY{o}{=} \PY{n}{sphere\PYZus{}function}\PY{p}{(}\PY{o}{*}\PY{n+nb+bp}{self}\PY{o}{.}\PY{n}{position}\PY{p}{)}
        \PY{k}{if} \PY{n}{current\PYZus{}score} \PY{o}{\PYZlt{}} \PY{n+nb+bp}{self}\PY{o}{.}\PY{n}{best\PYZus{}score}\PY{p}{:}
            \PY{n+nb+bp}{self}\PY{o}{.}\PY{n}{best\PYZus{}position} \PY{o}{=} \PY{n+nb+bp}{self}\PY{o}{.}\PY{n}{position}
            \PY{n+nb+bp}{self}\PY{o}{.}\PY{n}{best\PYZus{}score} \PY{o}{=} \PY{n}{current\PYZus{}score}
\end{Verbatim}
\end{tcolorbox}

    \hypertarget{funkcja-odpowiadajux105ca-za-wykonanie-algorytmu-pso}{%
\subsection{2.4 Funkcja odpowiadająca za wykonanie algorytmu
PSO}\label{funkcja-odpowiadajux105ca-za-wykonanie-algorytmu-pso}}

    \begin{tcolorbox}[breakable, size=fbox, boxrule=1pt, pad at break*=1mm,colback=cellbackground, colframe=cellborder]
\prompt{In}{incolor}{ }{\boxspacing}
\begin{Verbatim}[commandchars=\\\{\}]
\PY{k}{def} \PY{n+nf}{particle\PYZus{}swarm\PYZus{}optimization}\PY{p}{(}\PY{n}{fitness\PYZus{}function}\PY{p}{,} \PY{n}{num\PYZus{}particles}\PY{p}{,} \PY{n}{max\PYZus{}iterations}\PY{p}{)}\PY{p}{:}
    \PY{n}{swarm} \PY{o}{=} \PY{p}{[}\PY{n}{Particle}\PY{p}{(}\PY{p}{)} \PY{k}{for} \PY{n}{\PYZus{}} \PY{o+ow}{in} \PY{n+nb}{range}\PY{p}{(}\PY{n}{num\PYZus{}particles}\PY{p}{)}\PY{p}{]}
    \PY{n}{global\PYZus{}best\PYZus{}position} \PY{o}{=} \PY{n}{swarm}\PY{p}{[}\PY{l+m+mi}{0}\PY{p}{]}\PY{o}{.}\PY{n}{position}  \PY{c+c1}{\PYZsh{} Inicjalizacja globalnej najlepszej pozycji}
    \PY{n}{global\PYZus{}best\PYZus{}score} \PY{o}{=} \PY{n}{fitness\PYZus{}function}\PY{p}{(}\PY{o}{*}\PY{n}{global\PYZus{}best\PYZus{}position}\PY{p}{)}

    \PY{n}{positions} \PY{o}{=} \PY{p}{[}\PY{p}{]}  \PY{c+c1}{\PYZsh{} Lista pozycji cząstek w każdej iteracji}
    \PY{n}{fitness\PYZus{}value\PYZus{}swarm} \PY{o}{=} \PY{p}{[}\PY{p}{]} \PY{c+c1}{\PYZsh{} Lista wartości funkcji fitness w każdej iteracji dla populacji}
    \PY{n}{avr\PYZus{}fitness\PYZus{}value} \PY{o}{=} \PY{p}{[}\PY{p}{]} \PY{c+c1}{\PYZsh{} Lista wartości średniej wartości funkcji fitness dla cząsteczek}

    \PY{k}{for} \PY{n}{\PYZus{}} \PY{o+ow}{in} \PY{n+nb}{range}\PY{p}{(}\PY{n}{max\PYZus{}iterations}\PY{p}{)}\PY{p}{:}
        \PY{n}{iteration\PYZus{}positions} \PY{o}{=} \PY{p}{[}\PY{p}{]}  \PY{c+c1}{\PYZsh{} Pozycje cząstek w bieżącej iteracji}
        \PY{n}{sum\PYZus{}fitness\PYZus{}value} \PY{o}{=} \PY{l+m+mi}{0}
        \PY{k}{for} \PY{n}{particle} \PY{o+ow}{in} \PY{n}{swarm}\PY{p}{:}
            \PY{n}{particle}\PY{o}{.}\PY{n}{update\PYZus{}velocity}\PY{p}{(}\PY{n}{global\PYZus{}best\PYZus{}position}\PY{p}{,} \PY{l+m+mf}{0.5}\PY{p}{,} \PY{l+m+mf}{0.8}\PY{p}{,} \PY{l+m+mf}{0.8}\PY{p}{)}
            \PY{n}{particle}\PY{o}{.}\PY{n}{update\PYZus{}position}\PY{p}{(}\PY{p}{)}
            \PY{k}{if} \PY{n}{particle}\PY{o}{.}\PY{n}{best\PYZus{}score} \PY{o}{\PYZlt{}} \PY{n}{global\PYZus{}best\PYZus{}score}\PY{p}{:}
                \PY{n}{global\PYZus{}best\PYZus{}position} \PY{o}{=} \PY{n}{particle}\PY{o}{.}\PY{n}{best\PYZus{}position}
                \PY{n}{global\PYZus{}best\PYZus{}score} \PY{o}{=} \PY{n}{particle}\PY{o}{.}\PY{n}{best\PYZus{}score}
            \PY{n}{sum\PYZus{}fitness\PYZus{}value} \PY{o}{+}\PY{o}{=} \PY{n}{global\PYZus{}best\PYZus{}score}
            \PY{n}{iteration\PYZus{}positions}\PY{o}{.}\PY{n}{append}\PY{p}{(}\PY{n}{particle}\PY{o}{.}\PY{n}{position}\PY{p}{)}
        \PY{n}{positions}\PY{o}{.}\PY{n}{append}\PY{p}{(}\PY{n}{iteration\PYZus{}positions}\PY{p}{)}
        \PY{n}{avr\PYZus{}fitness\PYZus{}value}\PY{o}{.}\PY{n}{append}\PY{p}{(}\PY{n}{sum\PYZus{}fitness\PYZus{}value}\PY{o}{/}\PY{n+nb}{len}\PY{p}{(}\PY{n}{swarm}\PY{p}{)}\PY{p}{)}
        \PY{n}{fitness\PYZus{}value\PYZus{}swarm}\PY{o}{.}\PY{n}{append}\PY{p}{(}\PY{n}{global\PYZus{}best\PYZus{}score}\PY{p}{)}

    \PY{k}{return} \PY{n}{global\PYZus{}best\PYZus{}position}\PY{p}{,} \PY{n}{global\PYZus{}best\PYZus{}score}\PY{p}{,} \PY{n}{positions}\PY{p}{,} \PY{n}{fitness\PYZus{}value\PYZus{}swarm}\PY{p}{,} \PY{n}{avr\PYZus{}fitness\PYZus{}value}
\end{Verbatim}
\end{tcolorbox}

    \hypertarget{wywoux142ywanie-algorytmu-pso-wraz-z-odpowiednimi-parametrami-main}{%
\subsection{2.5 Wywoływanie algorytmu PSO wraz z odpowiednimi
parametrami
(MAIN)}\label{wywoux142ywanie-algorytmu-pso-wraz-z-odpowiednimi-parametrami-main}}

    \begin{tcolorbox}[breakable, size=fbox, boxrule=1pt, pad at break*=1mm,colback=cellbackground, colframe=cellborder]
\prompt{In}{incolor}{ }{\boxspacing}
\begin{Verbatim}[commandchars=\\\{\}]
\PY{c+c1}{\PYZsh{} Wywołanie algorytmu PSO dla funkcji sferycznej }
\PY{n}{best\PYZus{}position}\PY{p}{,} \PY{n}{best\PYZus{}score}\PY{p}{,} \PY{n}{positions}\PY{p}{,} \PY{n}{fitness\PYZus{}value\PYZus{}swarm}\PY{p}{,} \PY{n}{avr\PYZus{}fitness\PYZus{}value} \PY{o}{=} \PY{n}{particle\PYZus{}swarm\PYZus{}optimization}\PY{p}{(}\PY{n}{sphere\PYZus{}function}\PY{p}{,} \PY{n}{num\PYZus{}particles}\PY{o}{=}\PY{l+m+mi}{50}\PY{p}{,} \PY{n}{max\PYZus{}iterations}\PY{o}{=}\PY{l+m+mi}{100}\PY{p}{)}

\PY{c+c1}{\PYZsh{} Wywołanie algorytmu PSO dla funkcji rastrigin}
\PY{n}{best\PYZus{}position\PYZus{}r}\PY{p}{,} \PY{n}{best\PYZus{}score\PYZus{}r}\PY{p}{,} \PY{n}{positions\PYZus{}r}\PY{p}{,} \PY{n}{fitness\PYZus{}value\PYZus{}r\PYZus{}swarm}\PY{p}{,} \PY{n}{avr\PYZus{}fitness\PYZus{}value\PYZus{}r} \PY{o}{=} \PY{n}{particle\PYZus{}swarm\PYZus{}optimization}\PY{p}{(}\PY{n}{rastrigin\PYZus{}function}\PY{p}{,} \PY{n}{num\PYZus{}particles}\PY{o}{=}\PY{l+m+mi}{50}\PY{p}{,} \PY{n}{max\PYZus{}iterations}\PY{o}{=}\PY{l+m+mi}{100}\PY{p}{)}
\end{Verbatim}
\end{tcolorbox}

    \hypertarget{tworzenie-animacji-wykresu-ruchu-czux105stek-dla-funkcji-sferycznej}{%
\subsection{2.6 Tworzenie animacji wykresu ruchu cząstek dla funkcji
sferycznej}\label{tworzenie-animacji-wykresu-ruchu-czux105stek-dla-funkcji-sferycznej}}

    \begin{tcolorbox}[breakable, size=fbox, boxrule=1pt, pad at break*=1mm,colback=cellbackground, colframe=cellborder]
\prompt{In}{incolor}{ }{\boxspacing}
\begin{Verbatim}[commandchars=\\\{\}]
\PY{c+c1}{\PYZsh{} Tworzenie siatki punktów dla wykresu funkcji sferycznej}
\PY{n}{x} \PY{o}{=} \PY{n}{np}\PY{o}{.}\PY{n}{linspace}\PY{p}{(}\PY{o}{\PYZhy{}}\PY{l+m+mi}{10}\PY{p}{,} \PY{l+m+mi}{10}\PY{p}{,} \PY{l+m+mi}{100}\PY{p}{)}
\PY{n}{y} \PY{o}{=} \PY{n}{np}\PY{o}{.}\PY{n}{linspace}\PY{p}{(}\PY{o}{\PYZhy{}}\PY{l+m+mi}{10}\PY{p}{,} \PY{l+m+mi}{10}\PY{p}{,} \PY{l+m+mi}{100}\PY{p}{)}
\PY{n}{X}\PY{p}{,} \PY{n}{Y} \PY{o}{=} \PY{n}{np}\PY{o}{.}\PY{n}{meshgrid}\PY{p}{(}\PY{n}{x}\PY{p}{,} \PY{n}{y}\PY{p}{)}
\PY{n}{Z} \PY{o}{=} \PY{n}{sphere\PYZus{}function}\PY{p}{(}\PY{n}{X}\PY{p}{,} \PY{n}{Y}\PY{p}{)}

\PY{c+c1}{\PYZsh{} Inicjalizacja wykresu dla funkcji sferycznej}
\PY{n}{fig\PYZus{}sphere} \PY{o}{=} \PY{n}{plt}\PY{o}{.}\PY{n}{figure}\PY{p}{(}\PY{n}{figsize}\PY{o}{=}\PY{p}{(}\PY{l+m+mi}{8}\PY{p}{,} \PY{l+m+mi}{6}\PY{p}{)}\PY{p}{)}
\PY{n}{ax\PYZus{}sphere} \PY{o}{=} \PY{n}{fig\PYZus{}sphere}\PY{o}{.}\PY{n}{add\PYZus{}subplot}\PY{p}{(}\PY{l+m+mi}{111}\PY{p}{,} \PY{n}{projection}\PY{o}{=}\PY{l+s+s1}{\PYZsq{}}\PY{l+s+s1}{3d}\PY{l+s+s1}{\PYZsq{}}\PY{p}{)}

\PY{k}{def} \PY{n+nf}{animate\PYZus{}sphere}\PY{p}{(}\PY{n}{i}\PY{p}{)}\PY{p}{:}
    \PY{n}{ax\PYZus{}sphere}\PY{o}{.}\PY{n}{clear}\PY{p}{(}\PY{p}{)}
    \PY{n}{ax\PYZus{}sphere}\PY{o}{.}\PY{n}{set\PYZus{}title}\PY{p}{(}\PY{l+s+sa}{f}\PY{l+s+s1}{\PYZsq{}}\PY{l+s+s1}{Iteracja }\PY{l+s+si}{\PYZob{}}\PY{n}{i}\PY{o}{+}\PY{l+m+mi}{1}\PY{l+s+si}{\PYZcb{}}\PY{l+s+s1}{ \PYZhy{} Funkcja sferyczna}\PY{l+s+s1}{\PYZsq{}}\PY{p}{)}
    \PY{n}{ax\PYZus{}sphere}\PY{o}{.}\PY{n}{set\PYZus{}xlim}\PY{p}{(}\PY{o}{\PYZhy{}}\PY{l+m+mi}{10}\PY{p}{,} \PY{l+m+mi}{10}\PY{p}{)}
    \PY{n}{ax\PYZus{}sphere}\PY{o}{.}\PY{n}{set\PYZus{}ylim}\PY{p}{(}\PY{o}{\PYZhy{}}\PY{l+m+mi}{10}\PY{p}{,} \PY{l+m+mi}{10}\PY{p}{)}
    \PY{n}{ax\PYZus{}sphere}\PY{o}{.}\PY{n}{set\PYZus{}zlim}\PY{p}{(}\PY{l+m+mi}{0}\PY{p}{,} \PY{l+m+mi}{200}\PY{p}{)}
    \PY{n}{ax\PYZus{}sphere}\PY{o}{.}\PY{n}{set\PYZus{}xlabel}\PY{p}{(}\PY{l+s+s1}{\PYZsq{}}\PY{l+s+s1}{X}\PY{l+s+s1}{\PYZsq{}}\PY{p}{)}
    \PY{n}{ax\PYZus{}sphere}\PY{o}{.}\PY{n}{set\PYZus{}ylabel}\PY{p}{(}\PY{l+s+s1}{\PYZsq{}}\PY{l+s+s1}{Y}\PY{l+s+s1}{\PYZsq{}}\PY{p}{)}
    \PY{n}{ax\PYZus{}sphere}\PY{o}{.}\PY{n}{set\PYZus{}zlabel}\PY{p}{(}\PY{l+s+s1}{\PYZsq{}}\PY{l+s+s1}{Wartość funkcji}\PY{l+s+s1}{\PYZsq{}}\PY{p}{)}
    \PY{n}{ax\PYZus{}sphere}\PY{o}{.}\PY{n}{scatter}\PY{p}{(}\PY{p}{[}\PY{n}{p}\PY{p}{[}\PY{l+m+mi}{0}\PY{p}{]} \PY{k}{for} \PY{n}{p} \PY{o+ow}{in} \PY{n}{positions}\PY{p}{[}\PY{n}{i}\PY{p}{]}\PY{p}{]}\PY{p}{,} \PY{p}{[}\PY{n}{p}\PY{p}{[}\PY{l+m+mi}{1}\PY{p}{]} \PY{k}{for} \PY{n}{p} \PY{o+ow}{in} \PY{n}{positions}\PY{p}{[}\PY{n}{i}\PY{p}{]}\PY{p}{]}\PY{p}{,}
                  \PY{p}{[}\PY{n}{sphere\PYZus{}function}\PY{p}{(}\PY{o}{*}\PY{n}{p}\PY{p}{)} \PY{k}{for} \PY{n}{p} \PY{o+ow}{in} \PY{n}{positions}\PY{p}{[}\PY{n}{i}\PY{p}{]}\PY{p}{]}\PY{p}{,} \PY{n}{color}\PY{o}{=}\PY{l+s+s1}{\PYZsq{}}\PY{l+s+s1}{b}\PY{l+s+s1}{\PYZsq{}}\PY{p}{,} \PY{n}{s}\PY{o}{=}\PY{l+m+mi}{40}\PY{p}{,} \PY{n}{alpha}\PY{o}{=}\PY{l+m+mf}{1.0}\PY{p}{)}
    \PY{n}{ax\PYZus{}sphere}\PY{o}{.}\PY{n}{scatter}\PY{p}{(}\PY{n}{best\PYZus{}position}\PY{p}{[}\PY{l+m+mi}{0}\PY{p}{]}\PY{p}{,} \PY{n}{best\PYZus{}position}\PY{p}{[}\PY{l+m+mi}{1}\PY{p}{]}\PY{p}{,} \PY{n}{best\PYZus{}score}\PY{p}{,} \PY{n}{color}\PY{o}{=}\PY{l+s+s1}{\PYZsq{}}\PY{l+s+s1}{r}\PY{l+s+s1}{\PYZsq{}}\PY{p}{,} \PY{n}{marker}\PY{o}{=}\PY{l+s+s1}{\PYZsq{}}\PY{l+s+s1}{*}\PY{l+s+s1}{\PYZsq{}}\PY{p}{,} \PY{n}{s}\PY{o}{=}\PY{l+m+mi}{200}\PY{p}{,} \PY{n}{label}\PY{o}{=}\PY{l+s+s1}{\PYZsq{}}\PY{l+s+s1}{Najlepsza pozycja}\PY{l+s+s1}{\PYZsq{}}\PY{p}{)}
    \PY{n}{ax\PYZus{}sphere}\PY{o}{.}\PY{n}{plot\PYZus{}surface}\PY{p}{(}\PY{n}{X}\PY{p}{,} \PY{n}{Y}\PY{p}{,} \PY{n}{Z}\PY{p}{,} \PY{n}{cmap}\PY{o}{=}\PY{l+s+s1}{\PYZsq{}}\PY{l+s+s1}{viridis}\PY{l+s+s1}{\PYZsq{}}\PY{p}{,} \PY{n}{alpha}\PY{o}{=}\PY{l+m+mf}{0.5}\PY{p}{)}  \PY{c+c1}{\PYZsh{} Wyświetlanie funkcji sferycznej}
    \PY{n}{ax\PYZus{}sphere}\PY{o}{.}\PY{n}{legend}\PY{p}{(}\PY{p}{)}

\PY{n}{ani\PYZus{}sphere} \PY{o}{=} \PY{n}{FuncAnimation}\PY{p}{(}\PY{n}{fig\PYZus{}sphere}\PY{p}{,} \PY{n}{animate\PYZus{}sphere}\PY{p}{,} \PY{n}{frames}\PY{o}{=}\PY{n+nb}{len}\PY{p}{(}\PY{n}{positions}\PY{p}{)}\PY{p}{,} \PY{n}{interval}\PY{o}{=}\PY{l+m+mi}{600}\PY{p}{)}
\PY{n}{ani\PYZus{}sphere}\PY{o}{.}\PY{n}{save}\PY{p}{(}\PY{l+s+s1}{\PYZsq{}}\PY{l+s+s1}{animation\PYZus{}sphere.mp4}\PY{l+s+s1}{\PYZsq{}}\PY{p}{,} \PY{n}{writer}\PY{o}{=}\PY{l+s+s1}{\PYZsq{}}\PY{l+s+s1}{ffmpeg}\PY{l+s+s1}{\PYZsq{}}\PY{p}{,} \PY{n}{dpi}\PY{o}{=}\PY{l+m+mi}{100}\PY{p}{)}
\end{Verbatim}
\end{tcolorbox}

    \begin{center}
    \adjustimage{max size={0.9\linewidth}{0.9\paperheight}}{algorytmy_rojowe_files/algorytmy_rojowe_16_0.png}
    \end{center}
    { \hspace*{\fill} \\}
    
    \hypertarget{tworzenie-animacji-wykresu-ruchu-czux105stek-dla-funkcji-rastrigin}{%
\subsection{2.7 Tworzenie animacji wykresu ruchu cząstek dla funkcji
rastrigin}\label{tworzenie-animacji-wykresu-ruchu-czux105stek-dla-funkcji-rastrigin}}

    \begin{tcolorbox}[breakable, size=fbox, boxrule=1pt, pad at break*=1mm,colback=cellbackground, colframe=cellborder]
\prompt{In}{incolor}{ }{\boxspacing}
\begin{Verbatim}[commandchars=\\\{\}]
\PY{c+c1}{\PYZsh{} Tworzenie siatki punktów dla wykresu funkcji rastrigin}
\PY{n}{x\PYZus{}r} \PY{o}{=} \PY{n}{np}\PY{o}{.}\PY{n}{linspace}\PY{p}{(}\PY{o}{\PYZhy{}}\PY{l+m+mf}{5.12}\PY{p}{,} \PY{l+m+mf}{5.12}\PY{p}{,} \PY{l+m+mi}{100}\PY{p}{)}
\PY{n}{y\PYZus{}r} \PY{o}{=} \PY{n}{np}\PY{o}{.}\PY{n}{linspace}\PY{p}{(}\PY{o}{\PYZhy{}}\PY{l+m+mf}{5.12}\PY{p}{,} \PY{l+m+mf}{5.12}\PY{p}{,} \PY{l+m+mi}{100}\PY{p}{)}
\PY{n}{X\PYZus{}r}\PY{p}{,} \PY{n}{Y\PYZus{}r} \PY{o}{=} \PY{n}{np}\PY{o}{.}\PY{n}{meshgrid}\PY{p}{(}\PY{n}{x\PYZus{}r}\PY{p}{,} \PY{n}{y\PYZus{}r}\PY{p}{)}
\PY{n}{Z\PYZus{}r} \PY{o}{=} \PY{n}{rastrigin\PYZus{}function}\PY{p}{(}\PY{n}{X\PYZus{}r}\PY{p}{,} \PY{n}{Y\PYZus{}r}\PY{p}{)}

\PY{c+c1}{\PYZsh{} Inicjalizacja wykresu dla funkcji rastrigin}
\PY{n}{fig\PYZus{}rastrigin} \PY{o}{=} \PY{n}{plt}\PY{o}{.}\PY{n}{figure}\PY{p}{(}\PY{n}{figsize}\PY{o}{=}\PY{p}{(}\PY{l+m+mi}{8}\PY{p}{,} \PY{l+m+mi}{6}\PY{p}{)}\PY{p}{)}
\PY{n}{ax\PYZus{}rastrigin} \PY{o}{=} \PY{n}{fig\PYZus{}rastrigin}\PY{o}{.}\PY{n}{add\PYZus{}subplot}\PY{p}{(}\PY{l+m+mi}{111}\PY{p}{,} \PY{n}{projection}\PY{o}{=}\PY{l+s+s1}{\PYZsq{}}\PY{l+s+s1}{3d}\PY{l+s+s1}{\PYZsq{}}\PY{p}{)}

\PY{k}{def} \PY{n+nf}{animate\PYZus{}rastrigin}\PY{p}{(}\PY{n}{i}\PY{p}{)}\PY{p}{:}
    \PY{n}{ax\PYZus{}rastrigin}\PY{o}{.}\PY{n}{clear}\PY{p}{(}\PY{p}{)}
    \PY{n}{ax\PYZus{}rastrigin}\PY{o}{.}\PY{n}{set\PYZus{}title}\PY{p}{(}\PY{l+s+sa}{f}\PY{l+s+s1}{\PYZsq{}}\PY{l+s+s1}{Iteracja }\PY{l+s+si}{\PYZob{}}\PY{n}{i}\PY{o}{+}\PY{l+m+mi}{1}\PY{l+s+si}{\PYZcb{}}\PY{l+s+s1}{ \PYZhy{} Funkcja rastrigin}\PY{l+s+s1}{\PYZsq{}}\PY{p}{)}
    \PY{n}{ax\PYZus{}rastrigin}\PY{o}{.}\PY{n}{set\PYZus{}xlim}\PY{p}{(}\PY{o}{\PYZhy{}}\PY{l+m+mf}{5.12}\PY{p}{,} \PY{l+m+mf}{5.12}\PY{p}{)}
    \PY{n}{ax\PYZus{}rastrigin}\PY{o}{.}\PY{n}{set\PYZus{}ylim}\PY{p}{(}\PY{o}{\PYZhy{}}\PY{l+m+mf}{5.12}\PY{p}{,} \PY{l+m+mf}{5.12}\PY{p}{)}
    \PY{n}{ax\PYZus{}rastrigin}\PY{o}{.}\PY{n}{set\PYZus{}zlim}\PY{p}{(}\PY{l+m+mi}{0}\PY{p}{,} \PY{l+m+mi}{200}\PY{p}{)}
    \PY{n}{ax\PYZus{}rastrigin}\PY{o}{.}\PY{n}{set\PYZus{}xlabel}\PY{p}{(}\PY{l+s+s1}{\PYZsq{}}\PY{l+s+s1}{X}\PY{l+s+s1}{\PYZsq{}}\PY{p}{)}
    \PY{n}{ax\PYZus{}rastrigin}\PY{o}{.}\PY{n}{set\PYZus{}ylabel}\PY{p}{(}\PY{l+s+s1}{\PYZsq{}}\PY{l+s+s1}{Y}\PY{l+s+s1}{\PYZsq{}}\PY{p}{)}
    \PY{n}{ax\PYZus{}rastrigin}\PY{o}{.}\PY{n}{set\PYZus{}zlabel}\PY{p}{(}\PY{l+s+s1}{\PYZsq{}}\PY{l+s+s1}{Wartość funkcji}\PY{l+s+s1}{\PYZsq{}}\PY{p}{)}
    \PY{n}{ax\PYZus{}rastrigin}\PY{o}{.}\PY{n}{scatter}\PY{p}{(}\PY{p}{[}\PY{n}{p}\PY{p}{[}\PY{l+m+mi}{0}\PY{p}{]} \PY{k}{for} \PY{n}{p} \PY{o+ow}{in} \PY{n}{positions\PYZus{}r}\PY{p}{[}\PY{n}{i}\PY{p}{]}\PY{p}{]}\PY{p}{,} \PY{p}{[}\PY{n}{p}\PY{p}{[}\PY{l+m+mi}{1}\PY{p}{]} \PY{k}{for} \PY{n}{p} \PY{o+ow}{in} \PY{n}{positions\PYZus{}r}\PY{p}{[}\PY{n}{i}\PY{p}{]}\PY{p}{]}\PY{p}{,}
                     \PY{p}{[}\PY{n}{rastrigin\PYZus{}function}\PY{p}{(}\PY{o}{*}\PY{n}{p}\PY{p}{)} \PY{k}{for} \PY{n}{p} \PY{o+ow}{in} \PY{n}{positions\PYZus{}r}\PY{p}{[}\PY{n}{i}\PY{p}{]}\PY{p}{]}\PY{p}{,} \PY{n}{color}\PY{o}{=}\PY{l+s+s1}{\PYZsq{}}\PY{l+s+s1}{r}\PY{l+s+s1}{\PYZsq{}}\PY{p}{,} \PY{n}{s}\PY{o}{=}\PY{l+m+mi}{40}\PY{p}{,} \PY{n}{alpha}\PY{o}{=}\PY{l+m+mf}{1.0}\PY{p}{)}
    \PY{n}{ax\PYZus{}rastrigin}\PY{o}{.}\PY{n}{scatter}\PY{p}{(}\PY{n}{best\PYZus{}position\PYZus{}r}\PY{p}{[}\PY{l+m+mi}{0}\PY{p}{]}\PY{p}{,} \PY{n}{best\PYZus{}position\PYZus{}r}\PY{p}{[}\PY{l+m+mi}{1}\PY{p}{]}\PY{p}{,} \PY{n}{best\PYZus{}score\PYZus{}r}\PY{p}{,} \PY{n}{color}\PY{o}{=}\PY{l+s+s1}{\PYZsq{}}\PY{l+s+s1}{m}\PY{l+s+s1}{\PYZsq{}}\PY{p}{,} \PY{n}{marker}\PY{o}{=}\PY{l+s+s1}{\PYZsq{}}\PY{l+s+s1}{*}\PY{l+s+s1}{\PYZsq{}}\PY{p}{,} \PY{n}{s}\PY{o}{=}\PY{l+m+mi}{200}\PY{p}{,} \PY{n}{label}\PY{o}{=}\PY{l+s+s1}{\PYZsq{}}\PY{l+s+s1}{Najlepsza pozycja (Rastrigin)}\PY{l+s+s1}{\PYZsq{}}\PY{p}{)}
    \PY{n}{ax\PYZus{}rastrigin}\PY{o}{.}\PY{n}{plot\PYZus{}surface}\PY{p}{(}\PY{n}{X\PYZus{}r}\PY{p}{,} \PY{n}{Y\PYZus{}r}\PY{p}{,} \PY{n}{Z\PYZus{}r}\PY{p}{,} \PY{n}{cmap}\PY{o}{=}\PY{l+s+s1}{\PYZsq{}}\PY{l+s+s1}{viridis}\PY{l+s+s1}{\PYZsq{}}\PY{p}{,} \PY{n}{alpha}\PY{o}{=}\PY{l+m+mf}{0.5}\PY{p}{)}  \PY{c+c1}{\PYZsh{} Wyświetlanie funkcji rastrigin}
    \PY{n}{ax\PYZus{}rastrigin}\PY{o}{.}\PY{n}{legend}\PY{p}{(}\PY{p}{)}

\PY{n}{ani\PYZus{}rastrigin} \PY{o}{=} \PY{n}{FuncAnimation}\PY{p}{(}\PY{n}{fig\PYZus{}rastrigin}\PY{p}{,} \PY{n}{animate\PYZus{}rastrigin}\PY{p}{,} \PY{n}{frames}\PY{o}{=}\PY{n+nb}{len}\PY{p}{(}\PY{n}{positions\PYZus{}r}\PY{p}{)}\PY{p}{,} \PY{n}{interval}\PY{o}{=}\PY{l+m+mi}{600}\PY{p}{)}
\PY{n}{ani\PYZus{}rastrigin}\PY{o}{.}\PY{n}{save}\PY{p}{(}\PY{l+s+s1}{\PYZsq{}}\PY{l+s+s1}{animation\PYZus{}rastrigin.mp4}\PY{l+s+s1}{\PYZsq{}}\PY{p}{,} \PY{n}{writer}\PY{o}{=}\PY{l+s+s1}{\PYZsq{}}\PY{l+s+s1}{ffmpeg}\PY{l+s+s1}{\PYZsq{}}\PY{p}{,} \PY{n}{dpi}\PY{o}{=}\PY{l+m+mi}{100}\PY{p}{)}
\end{Verbatim}
\end{tcolorbox}

    \begin{center}
    \adjustimage{max size={0.9\linewidth}{0.9\paperheight}}{algorytmy_rojowe_files/algorytmy_rojowe_18_0.png}
    \end{center}
    { \hspace*{\fill} \\}
    
    \hypertarget{tworzenie-animacji-wykresu-ruchu-czux105stek-dla-funkcji-sferycznej-w-wersji-2d}{%
\subsection{2.8 Tworzenie animacji wykresu ruchu cząstek dla funkcji
sferycznej w wersji
2D}\label{tworzenie-animacji-wykresu-ruchu-czux105stek-dla-funkcji-sferycznej-w-wersji-2d}}

    \begin{tcolorbox}[breakable, size=fbox, boxrule=1pt, pad at break*=1mm,colback=cellbackground, colframe=cellborder]
\prompt{In}{incolor}{ }{\boxspacing}
\begin{Verbatim}[commandchars=\\\{\}]
\PY{c+c1}{\PYZsh{} Inicjalizacja wykresu dla funkcji sferycznej}
\PY{n}{fig\PYZus{}sphere} \PY{o}{=} \PY{n}{plt}\PY{o}{.}\PY{n}{figure}\PY{p}{(}\PY{n}{figsize}\PY{o}{=}\PY{p}{(}\PY{l+m+mi}{8}\PY{p}{,} \PY{l+m+mi}{6}\PY{p}{)}\PY{p}{)}
\PY{n}{ax\PYZus{}sphere} \PY{o}{=} \PY{n}{fig\PYZus{}sphere}\PY{o}{.}\PY{n}{add\PYZus{}subplot}\PY{p}{(}\PY{l+m+mi}{111}\PY{p}{)}

\PY{k}{def} \PY{n+nf}{animate\PYZus{}sphere}\PY{p}{(}\PY{n}{i}\PY{p}{)}\PY{p}{:}
    \PY{n}{ax\PYZus{}sphere}\PY{o}{.}\PY{n}{clear}\PY{p}{(}\PY{p}{)}
    \PY{n}{ax\PYZus{}sphere}\PY{o}{.}\PY{n}{set\PYZus{}title}\PY{p}{(}\PY{l+s+sa}{f}\PY{l+s+s1}{\PYZsq{}}\PY{l+s+s1}{Iteracja }\PY{l+s+si}{\PYZob{}}\PY{n}{i}\PY{o}{+}\PY{l+m+mi}{1}\PY{l+s+si}{\PYZcb{}}\PY{l+s+s1}{ \PYZhy{} Funkcja sferyczna}\PY{l+s+s1}{\PYZsq{}}\PY{p}{)}
    \PY{n}{ax\PYZus{}sphere}\PY{o}{.}\PY{n}{set\PYZus{}xlim}\PY{p}{(}\PY{o}{\PYZhy{}}\PY{l+m+mi}{10}\PY{p}{,} \PY{l+m+mi}{10}\PY{p}{)}
    \PY{n}{ax\PYZus{}sphere}\PY{o}{.}\PY{n}{set\PYZus{}ylim}\PY{p}{(}\PY{o}{\PYZhy{}}\PY{l+m+mi}{10}\PY{p}{,} \PY{l+m+mi}{10}\PY{p}{)}
    \PY{n}{ax\PYZus{}sphere}\PY{o}{.}\PY{n}{set\PYZus{}xlabel}\PY{p}{(}\PY{l+s+s1}{\PYZsq{}}\PY{l+s+s1}{X}\PY{l+s+s1}{\PYZsq{}}\PY{p}{)}
    \PY{n}{ax\PYZus{}sphere}\PY{o}{.}\PY{n}{set\PYZus{}ylabel}\PY{p}{(}\PY{l+s+s1}{\PYZsq{}}\PY{l+s+s1}{Y}\PY{l+s+s1}{\PYZsq{}}\PY{p}{)}
    \PY{n}{ax\PYZus{}sphere}\PY{o}{.}\PY{n}{scatter}\PY{p}{(}\PY{p}{[}\PY{n}{p}\PY{p}{[}\PY{l+m+mi}{0}\PY{p}{]} \PY{k}{for} \PY{n}{p} \PY{o+ow}{in} \PY{n}{positions}\PY{p}{[}\PY{n}{i}\PY{p}{]}\PY{p}{]}\PY{p}{,} \PY{p}{[}\PY{n}{p}\PY{p}{[}\PY{l+m+mi}{1}\PY{p}{]} \PY{k}{for} \PY{n}{p} \PY{o+ow}{in} \PY{n}{positions}\PY{p}{[}\PY{n}{i}\PY{p}{]}\PY{p}{]}\PY{p}{,} \PY{n}{color}\PY{o}{=}\PY{l+s+s1}{\PYZsq{}}\PY{l+s+s1}{b}\PY{l+s+s1}{\PYZsq{}}\PY{p}{,} \PY{n}{marker}\PY{o}{=}\PY{l+s+s1}{\PYZsq{}}\PY{l+s+s1}{o}\PY{l+s+s1}{\PYZsq{}}\PY{p}{)}
    \PY{n}{ax\PYZus{}sphere}\PY{o}{.}\PY{n}{scatter}\PY{p}{(}\PY{n}{best\PYZus{}position}\PY{p}{[}\PY{l+m+mi}{0}\PY{p}{]}\PY{p}{,} \PY{n}{best\PYZus{}position}\PY{p}{[}\PY{l+m+mi}{1}\PY{p}{]}\PY{p}{,} \PY{n}{color}\PY{o}{=}\PY{l+s+s1}{\PYZsq{}}\PY{l+s+s1}{r}\PY{l+s+s1}{\PYZsq{}}\PY{p}{,} \PY{n}{marker}\PY{o}{=}\PY{l+s+s1}{\PYZsq{}}\PY{l+s+s1}{*}\PY{l+s+s1}{\PYZsq{}}\PY{p}{,} \PY{n}{s}\PY{o}{=}\PY{l+m+mi}{200}\PY{p}{,} \PY{n}{label}\PY{o}{=}\PY{l+s+s1}{\PYZsq{}}\PY{l+s+s1}{Najlepsza pozycja}\PY{l+s+s1}{\PYZsq{}}\PY{p}{)}
    \PY{n}{ax\PYZus{}sphere}\PY{o}{.}\PY{n}{contourf}\PY{p}{(}\PY{n}{X}\PY{p}{,} \PY{n}{Y}\PY{p}{,} \PY{n}{Z}\PY{p}{,} \PY{n}{levels}\PY{o}{=}\PY{l+m+mi}{100}\PY{p}{,} \PY{n}{cmap}\PY{o}{=}\PY{l+s+s1}{\PYZsq{}}\PY{l+s+s1}{viridis}\PY{l+s+s1}{\PYZsq{}}\PY{p}{,} \PY{n}{alpha}\PY{o}{=}\PY{l+m+mf}{0.5}\PY{p}{)}  \PY{c+c1}{\PYZsh{} Wyświetlanie funkcji sferycznej}
    \PY{n}{ax\PYZus{}sphere}\PY{o}{.}\PY{n}{legend}\PY{p}{(}\PY{p}{)}

\PY{n}{ani\PYZus{}sphere} \PY{o}{=} \PY{n}{FuncAnimation}\PY{p}{(}\PY{n}{fig\PYZus{}sphere}\PY{p}{,} \PY{n}{animate\PYZus{}sphere}\PY{p}{,} \PY{n}{frames}\PY{o}{=}\PY{n+nb}{len}\PY{p}{(}\PY{n}{positions}\PY{p}{)}\PY{p}{,} \PY{n}{interval}\PY{o}{=}\PY{l+m+mi}{600}\PY{p}{)}
\PY{n}{ani\PYZus{}sphere}\PY{o}{.}\PY{n}{save}\PY{p}{(}\PY{l+s+s1}{\PYZsq{}}\PY{l+s+s1}{animation\PYZus{}sphere\PYZus{}top.mp4}\PY{l+s+s1}{\PYZsq{}}\PY{p}{,} \PY{n}{writer}\PY{o}{=}\PY{l+s+s1}{\PYZsq{}}\PY{l+s+s1}{ffmpeg}\PY{l+s+s1}{\PYZsq{}}\PY{p}{,} \PY{n}{dpi}\PY{o}{=}\PY{l+m+mi}{100}\PY{p}{)}
\end{Verbatim}
\end{tcolorbox}

    \begin{center}
    \adjustimage{max size={0.9\linewidth}{0.9\paperheight}}{algorytmy_rojowe_files/algorytmy_rojowe_20_0.png}
    \end{center}
    { \hspace*{\fill} \\}
    
    \hypertarget{tworzenie-animacji-wykresu-ruchu-czux105stek-dla-funkcji-rastrigin-w-wersji-2d}{%
\subsection{2.9 Tworzenie animacji wykresu ruchu cząstek dla funkcji
rastrigin w wersji
2D}\label{tworzenie-animacji-wykresu-ruchu-czux105stek-dla-funkcji-rastrigin-w-wersji-2d}}

    \begin{tcolorbox}[breakable, size=fbox, boxrule=1pt, pad at break*=1mm,colback=cellbackground, colframe=cellborder]
\prompt{In}{incolor}{ }{\boxspacing}
\begin{Verbatim}[commandchars=\\\{\}]
\PY{c+c1}{\PYZsh{} Tworzenie siatki punktów dla wykresu funkcji rastrigin}
\PY{n}{x\PYZus{}r} \PY{o}{=} \PY{n}{np}\PY{o}{.}\PY{n}{linspace}\PY{p}{(}\PY{o}{\PYZhy{}}\PY{l+m+mf}{5.12}\PY{p}{,} \PY{l+m+mf}{5.12}\PY{p}{,} \PY{l+m+mi}{100}\PY{p}{)}
\PY{n}{y\PYZus{}r} \PY{o}{=} \PY{n}{np}\PY{o}{.}\PY{n}{linspace}\PY{p}{(}\PY{o}{\PYZhy{}}\PY{l+m+mf}{5.12}\PY{p}{,} \PY{l+m+mf}{5.12}\PY{p}{,} \PY{l+m+mi}{100}\PY{p}{)}
\PY{n}{X\PYZus{}r}\PY{p}{,} \PY{n}{Y\PYZus{}r} \PY{o}{=} \PY{n}{np}\PY{o}{.}\PY{n}{meshgrid}\PY{p}{(}\PY{n}{x\PYZus{}r}\PY{p}{,} \PY{n}{y\PYZus{}r}\PY{p}{)}
\PY{n}{Z\PYZus{}r} \PY{o}{=} \PY{n}{rastrigin\PYZus{}function}\PY{p}{(}\PY{n}{X\PYZus{}r}\PY{p}{,} \PY{n}{Y\PYZus{}r}\PY{p}{)}

\PY{c+c1}{\PYZsh{} Inicjalizacja wykresu dla funkcji rastrigin}
\PY{n}{fig\PYZus{}rastrigin\PYZus{}top} \PY{o}{=} \PY{n}{plt}\PY{o}{.}\PY{n}{figure}\PY{p}{(}\PY{n}{figsize}\PY{o}{=}\PY{p}{(}\PY{l+m+mi}{8}\PY{p}{,} \PY{l+m+mi}{6}\PY{p}{)}\PY{p}{)}
\PY{n}{ax\PYZus{}rastrigin\PYZus{}top} \PY{o}{=} \PY{n}{fig\PYZus{}rastrigin\PYZus{}top}\PY{o}{.}\PY{n}{add\PYZus{}subplot}\PY{p}{(}\PY{l+m+mi}{111}\PY{p}{)}

\PY{k}{def} \PY{n+nf}{animate\PYZus{}rastrigin\PYZus{}top}\PY{p}{(}\PY{n}{i}\PY{p}{)}\PY{p}{:}
    \PY{n}{ax\PYZus{}rastrigin\PYZus{}top}\PY{o}{.}\PY{n}{clear}\PY{p}{(}\PY{p}{)}
    \PY{n}{ax\PYZus{}rastrigin\PYZus{}top}\PY{o}{.}\PY{n}{set\PYZus{}title}\PY{p}{(}\PY{l+s+sa}{f}\PY{l+s+s1}{\PYZsq{}}\PY{l+s+s1}{Iteracja }\PY{l+s+si}{\PYZob{}}\PY{n}{i}\PY{o}{+}\PY{l+m+mi}{1}\PY{l+s+si}{\PYZcb{}}\PY{l+s+s1}{ \PYZhy{} Funkcja rastrigin}\PY{l+s+s1}{\PYZsq{}}\PY{p}{)}
    \PY{n}{ax\PYZus{}rastrigin\PYZus{}top}\PY{o}{.}\PY{n}{set\PYZus{}xlim}\PY{p}{(}\PY{o}{\PYZhy{}}\PY{l+m+mf}{5.12}\PY{p}{,} \PY{l+m+mf}{5.12}\PY{p}{)}
    \PY{n}{ax\PYZus{}rastrigin\PYZus{}top}\PY{o}{.}\PY{n}{set\PYZus{}ylim}\PY{p}{(}\PY{o}{\PYZhy{}}\PY{l+m+mf}{5.12}\PY{p}{,} \PY{l+m+mf}{5.12}\PY{p}{)}
    \PY{n}{ax\PYZus{}rastrigin\PYZus{}top}\PY{o}{.}\PY{n}{set\PYZus{}xlabel}\PY{p}{(}\PY{l+s+s1}{\PYZsq{}}\PY{l+s+s1}{X}\PY{l+s+s1}{\PYZsq{}}\PY{p}{)}
    \PY{n}{ax\PYZus{}rastrigin\PYZus{}top}\PY{o}{.}\PY{n}{set\PYZus{}ylabel}\PY{p}{(}\PY{l+s+s1}{\PYZsq{}}\PY{l+s+s1}{Y}\PY{l+s+s1}{\PYZsq{}}\PY{p}{)}
    \PY{n}{ax\PYZus{}rastrigin\PYZus{}top}\PY{o}{.}\PY{n}{contourf}\PY{p}{(}\PY{n}{X\PYZus{}r}\PY{p}{,} \PY{n}{Y\PYZus{}r}\PY{p}{,} \PY{n}{Z\PYZus{}r}\PY{p}{,} \PY{n}{levels}\PY{o}{=}\PY{l+m+mi}{100}\PY{p}{,} \PY{n}{cmap}\PY{o}{=}\PY{l+s+s1}{\PYZsq{}}\PY{l+s+s1}{viridis}\PY{l+s+s1}{\PYZsq{}}\PY{p}{,} \PY{n}{alpha}\PY{o}{=}\PY{l+m+mf}{0.5}\PY{p}{)}  \PY{c+c1}{\PYZsh{} Wyświetlanie funkcji rastrigin}
    \PY{n}{ax\PYZus{}rastrigin\PYZus{}top}\PY{o}{.}\PY{n}{scatter}\PY{p}{(}\PY{p}{[}\PY{n}{p}\PY{p}{[}\PY{l+m+mi}{0}\PY{p}{]} \PY{k}{for} \PY{n}{p} \PY{o+ow}{in} \PY{n}{positions\PYZus{}r}\PY{p}{[}\PY{n}{i}\PY{p}{]}\PY{p}{]}\PY{p}{,} \PY{p}{[}\PY{n}{p}\PY{p}{[}\PY{l+m+mi}{1}\PY{p}{]} \PY{k}{for} \PY{n}{p} \PY{o+ow}{in} \PY{n}{positions\PYZus{}r}\PY{p}{[}\PY{n}{i}\PY{p}{]}\PY{p}{]}\PY{p}{,} \PY{n}{color}\PY{o}{=}\PY{l+s+s1}{\PYZsq{}}\PY{l+s+s1}{r}\PY{l+s+s1}{\PYZsq{}}\PY{p}{,} \PY{n}{marker}\PY{o}{=}\PY{l+s+s1}{\PYZsq{}}\PY{l+s+s1}{o}\PY{l+s+s1}{\PYZsq{}}\PY{p}{)}
    \PY{n}{ax\PYZus{}rastrigin\PYZus{}top}\PY{o}{.}\PY{n}{scatter}\PY{p}{(}\PY{n}{best\PYZus{}position\PYZus{}r}\PY{p}{[}\PY{l+m+mi}{0}\PY{p}{]}\PY{p}{,} \PY{n}{best\PYZus{}position\PYZus{}r}\PY{p}{[}\PY{l+m+mi}{1}\PY{p}{]}\PY{p}{,} \PY{n}{color}\PY{o}{=}\PY{l+s+s1}{\PYZsq{}}\PY{l+s+s1}{m}\PY{l+s+s1}{\PYZsq{}}\PY{p}{,} \PY{n}{marker}\PY{o}{=}\PY{l+s+s1}{\PYZsq{}}\PY{l+s+s1}{*}\PY{l+s+s1}{\PYZsq{}}\PY{p}{,} \PY{n}{s}\PY{o}{=}\PY{l+m+mi}{200}\PY{p}{,} \PY{n}{label}\PY{o}{=}\PY{l+s+s1}{\PYZsq{}}\PY{l+s+s1}{Najlepsza pozycja (Rastrigin)}\PY{l+s+s1}{\PYZsq{}}\PY{p}{)}
    \PY{n}{ax\PYZus{}rastrigin\PYZus{}top}\PY{o}{.}\PY{n}{legend}\PY{p}{(}\PY{p}{)}

\PY{n}{ani\PYZus{}rastrigin\PYZus{}top} \PY{o}{=} \PY{n}{FuncAnimation}\PY{p}{(}\PY{n}{fig\PYZus{}rastrigin\PYZus{}top}\PY{p}{,} \PY{n}{animate\PYZus{}rastrigin\PYZus{}top}\PY{p}{,} \PY{n}{frames}\PY{o}{=}\PY{n+nb}{len}\PY{p}{(}\PY{n}{positions\PYZus{}r}\PY{p}{)}\PY{p}{,} \PY{n}{interval}\PY{o}{=}\PY{l+m+mi}{600}\PY{p}{)}
\PY{n}{ani\PYZus{}rastrigin\PYZus{}top}\PY{o}{.}\PY{n}{save}\PY{p}{(}\PY{l+s+s1}{\PYZsq{}}\PY{l+s+s1}{animation\PYZus{}rastrigin\PYZus{}top.mp4}\PY{l+s+s1}{\PYZsq{}}\PY{p}{,} \PY{n}{writer}\PY{o}{=}\PY{l+s+s1}{\PYZsq{}}\PY{l+s+s1}{ffmpeg}\PY{l+s+s1}{\PYZsq{}}\PY{p}{,} \PY{n}{dpi}\PY{o}{=}\PY{l+m+mi}{100}\PY{p}{)}
\end{Verbatim}
\end{tcolorbox}

    \begin{center}
    \adjustimage{max size={0.9\linewidth}{0.9\paperheight}}{algorytmy_rojowe_files/algorytmy_rojowe_22_0.png}
    \end{center}
    { \hspace*{\fill} \\}
    
    \hypertarget{wykres-zmiany-wartoux15bci-funkcji-fitness-wzglux119dem-kaux17cdej-iteracji-dla-przykux142adu-funkcji-sferycznej}{%
\subsection{2.10 Wykres zmiany wartości funkcji fitness względem każdej
iteracji dla przykładu funkcji
sferycznej}\label{wykres-zmiany-wartoux15bci-funkcji-fitness-wzglux119dem-kaux17cdej-iteracji-dla-przykux142adu-funkcji-sferycznej}}

    \begin{tcolorbox}[breakable, size=fbox, boxrule=1pt, pad at break*=1mm,colback=cellbackground, colframe=cellborder]
\prompt{In}{incolor}{ }{\boxspacing}
\begin{Verbatim}[commandchars=\\\{\}]
\PY{c+c1}{\PYZsh{} Inicjalizacja wykresu dla funkcji sphere}
\PY{n}{fig\PYZus{}fitness\PYZus{}sphere} \PY{o}{=} \PY{n}{plt}\PY{o}{.}\PY{n}{figure}\PY{p}{(}\PY{n}{figsize}\PY{o}{=}\PY{p}{(}\PY{l+m+mi}{15}\PY{p}{,} \PY{l+m+mi}{6}\PY{p}{)}\PY{p}{)}
\PY{n}{ax\PYZus{}fitness\PYZus{}sphere} \PY{o}{=} \PY{n}{fig\PYZus{}fitness\PYZus{}sphere}\PY{o}{.}\PY{n}{add\PYZus{}subplot}\PY{p}{(}\PY{l+m+mi}{111}\PY{p}{)}

\PY{n}{itteration} \PY{o}{=} \PY{p}{[}\PY{n}{itt} \PY{k}{for} \PY{n}{itt} \PY{o+ow}{in} \PY{n+nb}{range}\PY{p}{(}\PY{l+m+mi}{0}\PY{p}{,} \PY{n+nb}{len}\PY{p}{(}\PY{n}{fitness\PYZus{}value\PYZus{}swarm}\PY{p}{)}\PY{p}{)}\PY{p}{]}

\PY{k}{def} \PY{n+nf}{animate\PYZus{}fitness\PYZus{}sphere}\PY{p}{(}\PY{n}{i}\PY{p}{)}\PY{p}{:}
    \PY{n}{ax\PYZus{}fitness\PYZus{}sphere}\PY{o}{.}\PY{n}{clear}\PY{p}{(}\PY{p}{)}
    \PY{n}{ax\PYZus{}fitness\PYZus{}sphere}\PY{o}{.}\PY{n}{set\PYZus{}title}\PY{p}{(}\PY{l+s+s1}{\PYZsq{}}\PY{l+s+s1}{Wykres wartości funkcji fitness dla przypadku funkcji sferycznej względem itteracji}\PY{l+s+s1}{\PYZsq{}}\PY{p}{)}
    \PY{n}{ax\PYZus{}fitness\PYZus{}sphere}\PY{o}{.}\PY{n}{set\PYZus{}xticks}\PY{p}{(}\PY{n}{np}\PY{o}{.}\PY{n}{arange}\PY{p}{(}\PY{l+m+mi}{0}\PY{p}{,} \PY{n+nb}{len}\PY{p}{(}\PY{n}{fitness\PYZus{}value\PYZus{}swarm}\PY{p}{)}\PY{p}{,} \PY{l+m+mi}{2}\PY{p}{)}\PY{p}{)}
    \PY{n}{ax\PYZus{}fitness\PYZus{}sphere}\PY{o}{.}\PY{n}{set\PYZus{}xlim}\PY{p}{(}\PY{l+m+mi}{0}\PY{p}{,} \PY{l+m+mi}{100}\PY{p}{)}
    \PY{n}{ax\PYZus{}fitness\PYZus{}sphere}\PY{o}{.}\PY{n}{set\PYZus{}ylim}\PY{p}{(}\PY{l+m+mi}{0}\PY{p}{,} \PY{l+m+mi}{1}\PY{p}{)}
    \PY{n}{ax\PYZus{}fitness\PYZus{}sphere}\PY{o}{.}\PY{n}{set\PYZus{}xlabel}\PY{p}{(}\PY{l+s+s1}{\PYZsq{}}\PY{l+s+s1}{Numer iteracji}\PY{l+s+s1}{\PYZsq{}}\PY{p}{)}
    \PY{n}{ax\PYZus{}fitness\PYZus{}sphere}\PY{o}{.}\PY{n}{set\PYZus{}ylabel}\PY{p}{(}\PY{l+s+s1}{\PYZsq{}}\PY{l+s+s1}{Wartość funkcji}\PY{l+s+s1}{\PYZsq{}}\PY{p}{)}
    \PY{n}{data} \PY{o}{=} \PY{n+nb}{list}\PY{p}{(}\PY{n+nb}{zip}\PY{p}{(}\PY{n}{itteration}\PY{p}{,} \PY{n}{fitness\PYZus{}value\PYZus{}swarm}\PY{p}{[}\PY{l+m+mi}{0}\PY{p}{:}\PY{n}{i}\PY{o}{+}\PY{l+m+mi}{1}\PY{p}{]}\PY{p}{)}\PY{p}{)}
    \PY{n}{ax\PYZus{}fitness\PYZus{}sphere}\PY{o}{.}\PY{n}{plot}\PY{p}{(}\PY{o}{*}\PY{n+nb}{zip}\PY{p}{(}\PY{o}{*}\PY{n}{data}\PY{p}{)}\PY{p}{,} \PY{n}{color}\PY{o}{=}\PY{l+s+s1}{\PYZsq{}}\PY{l+s+s1}{r}\PY{l+s+s1}{\PYZsq{}}\PY{p}{)}

\PY{n}{ani\PYZus{}fitness\PYZus{}sphere} \PY{o}{=} \PY{n}{FuncAnimation}\PY{p}{(}\PY{n}{fig\PYZus{}fitness\PYZus{}sphere}\PY{p}{,} \PY{n}{animate\PYZus{}fitness\PYZus{}sphere}\PY{p}{,} \PY{n}{frames}\PY{o}{=}\PY{n+nb}{len}\PY{p}{(}\PY{n}{fitness\PYZus{}value\PYZus{}swarm}\PY{p}{)}\PY{p}{,} \PY{n}{init\PYZus{}func}\PY{o}{=}\PY{k}{lambda}\PY{p}{:} \PY{k+kc}{None}\PY{p}{,} \PY{n}{interval}\PY{o}{=}\PY{l+m+mi}{600}\PY{p}{)}
\PY{n}{ani\PYZus{}fitness\PYZus{}sphere}\PY{o}{.}\PY{n}{save}\PY{p}{(}\PY{l+s+s1}{\PYZsq{}}\PY{l+s+s1}{animation\PYZus{}fitness\PYZus{}sphere.gif}\PY{l+s+s1}{\PYZsq{}}\PY{p}{,} \PY{n}{writer}\PY{o}{=}\PY{l+s+s1}{\PYZsq{}}\PY{l+s+s1}{Pillow}\PY{l+s+s1}{\PYZsq{}}\PY{p}{,} \PY{n}{dpi}\PY{o}{=}\PY{l+m+mi}{100}\PY{p}{)}
\end{Verbatim}
\end{tcolorbox}

    \begin{Verbatim}[commandchars=\\\{\}]
MovieWriter Pillow unavailable; using Pillow instead.
    \end{Verbatim}

    \begin{center}
    \adjustimage{max size={0.9\linewidth}{0.9\paperheight}}{algorytmy_rojowe_files/algorytmy_rojowe_24_1.png}
    \end{center}
    { \hspace*{\fill} \\}
    
    \hypertarget{wykres-zmiany-wartoux15bci-funkcji-fitness-wzglux119dem-kaux17cdej-iteracji-dla-przykux142adu-funkcji-rastrigin}{%
\subsection{2.11 Wykres zmiany wartości funkcji fitness względem każdej
iteracji dla przykładu funkcji
rastrigin}\label{wykres-zmiany-wartoux15bci-funkcji-fitness-wzglux119dem-kaux17cdej-iteracji-dla-przykux142adu-funkcji-rastrigin}}

    \begin{tcolorbox}[breakable, size=fbox, boxrule=1pt, pad at break*=1mm,colback=cellbackground, colframe=cellborder]
\prompt{In}{incolor}{ }{\boxspacing}
\begin{Verbatim}[commandchars=\\\{\}]
\PY{c+c1}{\PYZsh{} Inicjalizacja wykresu dla funkcji rastrigin}
\PY{n}{fig\PYZus{}fitness\PYZus{}rastrigin} \PY{o}{=} \PY{n}{plt}\PY{o}{.}\PY{n}{figure}\PY{p}{(}\PY{n}{figsize}\PY{o}{=}\PY{p}{(}\PY{l+m+mi}{15}\PY{p}{,} \PY{l+m+mi}{6}\PY{p}{)}\PY{p}{)}
\PY{n}{ax\PYZus{}fitness\PYZus{}rastrigin} \PY{o}{=} \PY{n}{fig\PYZus{}fitness\PYZus{}rastrigin}\PY{o}{.}\PY{n}{add\PYZus{}subplot}\PY{p}{(}\PY{l+m+mi}{111}\PY{p}{)}

\PY{n}{itteration} \PY{o}{=} \PY{p}{[}\PY{n}{itt} \PY{k}{for} \PY{n}{itt} \PY{o+ow}{in} \PY{n+nb}{range}\PY{p}{(}\PY{l+m+mi}{0}\PY{p}{,} \PY{n+nb}{len}\PY{p}{(}\PY{n}{fitness\PYZus{}value\PYZus{}r\PYZus{}swarm}\PY{p}{)}\PY{p}{)}\PY{p}{]}

\PY{k}{def} \PY{n+nf}{animate\PYZus{}fitness\PYZus{}rastrigin}\PY{p}{(}\PY{n}{i}\PY{p}{)}\PY{p}{:}
    \PY{n}{ax\PYZus{}fitness\PYZus{}rastrigin}\PY{o}{.}\PY{n}{clear}\PY{p}{(}\PY{p}{)}
    \PY{n}{ax\PYZus{}fitness\PYZus{}rastrigin}\PY{o}{.}\PY{n}{set\PYZus{}title}\PY{p}{(}\PY{l+s+s1}{\PYZsq{}}\PY{l+s+s1}{Wykres wartości funkcji fitness dla przypadku funkcji rastrigin względem itteracji}\PY{l+s+s1}{\PYZsq{}}\PY{p}{)}
    \PY{n}{ax\PYZus{}fitness\PYZus{}rastrigin}\PY{o}{.}\PY{n}{set\PYZus{}xticks}\PY{p}{(}\PY{n}{np}\PY{o}{.}\PY{n}{arange}\PY{p}{(}\PY{l+m+mi}{0}\PY{p}{,} \PY{n+nb}{len}\PY{p}{(}\PY{n}{fitness\PYZus{}value\PYZus{}r\PYZus{}swarm}\PY{p}{)}\PY{p}{,} \PY{l+m+mi}{2}\PY{p}{)}\PY{p}{)}
    \PY{n}{ax\PYZus{}fitness\PYZus{}rastrigin}\PY{o}{.}\PY{n}{set\PYZus{}xlim}\PY{p}{(}\PY{l+m+mi}{0}\PY{p}{,}\PY{l+m+mi}{100}\PY{p}{)}
    \PY{n}{ax\PYZus{}fitness\PYZus{}rastrigin}\PY{o}{.}\PY{n}{set\PYZus{}ylim}\PY{p}{(}\PY{l+m+mi}{0}\PY{p}{,} \PY{l+m+mi}{1}\PY{p}{)}
    \PY{n}{ax\PYZus{}fitness\PYZus{}rastrigin}\PY{o}{.}\PY{n}{set\PYZus{}xlabel}\PY{p}{(}\PY{l+s+s1}{\PYZsq{}}\PY{l+s+s1}{Numer iteracji}\PY{l+s+s1}{\PYZsq{}}\PY{p}{)}
    \PY{n}{ax\PYZus{}fitness\PYZus{}rastrigin}\PY{o}{.}\PY{n}{set\PYZus{}ylabel}\PY{p}{(}\PY{l+s+s1}{\PYZsq{}}\PY{l+s+s1}{Wartość funkcji}\PY{l+s+s1}{\PYZsq{}}\PY{p}{)}
    \PY{c+c1}{\PYZsh{} zip value to plot bcs feedback was error about diffrent shape of arrays}
    \PY{n}{data} \PY{o}{=} \PY{n+nb}{list}\PY{p}{(}\PY{n+nb}{zip}\PY{p}{(}\PY{n}{itteration}\PY{p}{,} \PY{n}{fitness\PYZus{}value\PYZus{}r\PYZus{}swarm}\PY{p}{[}\PY{l+m+mi}{0}\PY{p}{:}\PY{n}{i}\PY{o}{+}\PY{l+m+mi}{1}\PY{p}{]}\PY{p}{)}\PY{p}{)}
    \PY{n}{ax\PYZus{}fitness\PYZus{}rastrigin}\PY{o}{.}\PY{n}{plot}\PY{p}{(}\PY{o}{*}\PY{n+nb}{zip}\PY{p}{(}\PY{o}{*}\PY{n}{data}\PY{p}{)}\PY{p}{,} \PY{n}{color}\PY{o}{=}\PY{l+s+s1}{\PYZsq{}}\PY{l+s+s1}{r}\PY{l+s+s1}{\PYZsq{}}\PY{p}{)}

\PY{n}{ani\PYZus{}fitness\PYZus{}rastrigin} \PY{o}{=} \PY{n}{FuncAnimation}\PY{p}{(}\PY{n}{fig\PYZus{}fitness\PYZus{}rastrigin}\PY{p}{,} \PY{n}{animate\PYZus{}fitness\PYZus{}rastrigin}\PY{p}{,} \PY{n}{frames}\PY{o}{=}\PY{n+nb}{len}\PY{p}{(}\PY{n}{fitness\PYZus{}value\PYZus{}r\PYZus{}swarm}\PY{p}{)}\PY{p}{,} \PY{n}{init\PYZus{}func}\PY{o}{=}\PY{k}{lambda}\PY{p}{:} \PY{k+kc}{None}\PY{p}{,} \PY{n}{interval}\PY{o}{=}\PY{l+m+mi}{600}\PY{p}{)}
\PY{n}{ani\PYZus{}fitness\PYZus{}rastrigin}\PY{o}{.}\PY{n}{save}\PY{p}{(}\PY{l+s+s1}{\PYZsq{}}\PY{l+s+s1}{animation\PYZus{}fitness\PYZus{}rastrigin.gif}\PY{l+s+s1}{\PYZsq{}}\PY{p}{,} \PY{n}{writer}\PY{o}{=}\PY{l+s+s1}{\PYZsq{}}\PY{l+s+s1}{Pillow}\PY{l+s+s1}{\PYZsq{}}\PY{p}{,} \PY{n}{dpi}\PY{o}{=}\PY{l+m+mi}{100}\PY{p}{)}
\end{Verbatim}
\end{tcolorbox}

    \begin{Verbatim}[commandchars=\\\{\}]
MovieWriter Pillow unavailable; using Pillow instead.
    \end{Verbatim}

    \begin{center}
    \adjustimage{max size={0.9\linewidth}{0.9\paperheight}}{algorytmy_rojowe_files/algorytmy_rojowe_26_1.png}
    \end{center}
    { \hspace*{\fill} \\}
    
    \hypertarget{wykres-ux15bredniej-wartoux15bci-funkcji-celu-dla-kaux17cdej-czux105steczki-dla-funkcji-sferycznej}{%
\subsection{2.12 Wykres średniej wartości funkcji celu dla każdej
cząsteczki dla funkcji
sferycznej}\label{wykres-ux15bredniej-wartoux15bci-funkcji-celu-dla-kaux17cdej-czux105steczki-dla-funkcji-sferycznej}}

    \begin{tcolorbox}[breakable, size=fbox, boxrule=1pt, pad at break*=1mm,colback=cellbackground, colframe=cellborder]
\prompt{In}{incolor}{ }{\boxspacing}
\begin{Verbatim}[commandchars=\\\{\}]
\PY{c+c1}{\PYZsh{} Inicjalizacja wykresu dla funkcji sphere}
\PY{n}{fig\PYZus{}avr\PYZus{}fitness} \PY{o}{=} \PY{n}{plt}\PY{o}{.}\PY{n}{figure}\PY{p}{(}\PY{n}{figsize}\PY{o}{=}\PY{p}{(}\PY{l+m+mi}{15}\PY{p}{,} \PY{l+m+mi}{6}\PY{p}{)}\PY{p}{)}
\PY{n}{ax\PYZus{}avr\PYZus{}fitness} \PY{o}{=} \PY{n}{fig\PYZus{}avr\PYZus{}fitness}\PY{o}{.}\PY{n}{add\PYZus{}subplot}\PY{p}{(}\PY{l+m+mi}{111}\PY{p}{)}

\PY{n}{itteration} \PY{o}{=} \PY{p}{[}\PY{n}{itt} \PY{k}{for} \PY{n}{itt} \PY{o+ow}{in} \PY{n+nb}{range}\PY{p}{(}\PY{l+m+mi}{0}\PY{p}{,} \PY{n+nb}{len}\PY{p}{(}\PY{n}{avr\PYZus{}fitness\PYZus{}value}\PY{p}{)}\PY{p}{)}\PY{p}{]}

\PY{c+c1}{\PYZsh{} print(avr\PYZus{}fitness\PYZus{}value)}

\PY{k}{def} \PY{n+nf}{animate\PYZus{}avr\PYZus{}fitness}\PY{p}{(}\PY{n}{i}\PY{p}{)}\PY{p}{:}
    \PY{n}{ax\PYZus{}avr\PYZus{}fitness}\PY{o}{.}\PY{n}{clear}\PY{p}{(}\PY{p}{)}
    \PY{n}{ax\PYZus{}avr\PYZus{}fitness}\PY{o}{.}\PY{n}{set\PYZus{}title}\PY{p}{(}\PY{l+s+s1}{\PYZsq{}}\PY{l+s+s1}{Wykres wartości średniej funkcji fitness sphere względem każdej iteracji}\PY{l+s+s1}{\PYZsq{}}\PY{p}{)}
    \PY{n}{ax\PYZus{}avr\PYZus{}fitness}\PY{o}{.}\PY{n}{set\PYZus{}xticks}\PY{p}{(}\PY{n}{np}\PY{o}{.}\PY{n}{arange}\PY{p}{(}\PY{l+m+mi}{0}\PY{p}{,} \PY{n+nb}{len}\PY{p}{(}\PY{n}{avr\PYZus{}fitness\PYZus{}value}\PY{p}{)}\PY{p}{,} \PY{l+m+mi}{2}\PY{p}{)}\PY{p}{)}
    \PY{n}{ax\PYZus{}avr\PYZus{}fitness}\PY{o}{.}\PY{n}{set\PYZus{}xlim}\PY{p}{(}\PY{l+m+mi}{0}\PY{p}{,} \PY{l+m+mi}{100}\PY{p}{)}
    \PY{n}{ax\PYZus{}avr\PYZus{}fitness}\PY{o}{.}\PY{n}{set\PYZus{}ylim}\PY{p}{(}\PY{l+m+mi}{0}\PY{p}{,} \PY{l+m+mi}{1}\PY{p}{)}
    \PY{n}{ax\PYZus{}avr\PYZus{}fitness}\PY{o}{.}\PY{n}{set\PYZus{}xlabel}\PY{p}{(}\PY{l+s+s1}{\PYZsq{}}\PY{l+s+s1}{Numer iteracji}\PY{l+s+s1}{\PYZsq{}}\PY{p}{)}
    \PY{n}{ax\PYZus{}avr\PYZus{}fitness}\PY{o}{.}\PY{n}{set\PYZus{}ylabel}\PY{p}{(}\PY{l+s+s1}{\PYZsq{}}\PY{l+s+s1}{Średnia wartość funkcji}\PY{l+s+s1}{\PYZsq{}}\PY{p}{)}
    \PY{n}{data} \PY{o}{=} \PY{n+nb}{list}\PY{p}{(}\PY{n+nb}{zip}\PY{p}{(}\PY{n}{itteration}\PY{p}{,} \PY{n}{avr\PYZus{}fitness\PYZus{}value}\PY{p}{[}\PY{l+m+mi}{0}\PY{p}{:}\PY{n}{i}\PY{o}{+}\PY{l+m+mi}{1}\PY{p}{]}\PY{p}{)}\PY{p}{)}
    \PY{n}{ax\PYZus{}avr\PYZus{}fitness}\PY{o}{.}\PY{n}{plot}\PY{p}{(}\PY{o}{*}\PY{n+nb}{zip}\PY{p}{(}\PY{o}{*}\PY{n}{data}\PY{p}{)}\PY{p}{,} \PY{n}{color}\PY{o}{=}\PY{l+s+s1}{\PYZsq{}}\PY{l+s+s1}{r}\PY{l+s+s1}{\PYZsq{}}\PY{p}{)}

\PY{n}{ani\PYZus{}avr\PYZus{}fitness} \PY{o}{=} \PY{n}{FuncAnimation}\PY{p}{(}\PY{n}{fig\PYZus{}avr\PYZus{}fitness}\PY{p}{,} \PY{n}{animate\PYZus{}avr\PYZus{}fitness}\PY{p}{,} \PY{n}{frames}\PY{o}{=}\PY{n+nb}{len}\PY{p}{(}\PY{n}{avr\PYZus{}fitness\PYZus{}value}\PY{p}{)}\PY{p}{,} \PY{n}{init\PYZus{}func}\PY{o}{=}\PY{k}{lambda}\PY{p}{:} \PY{k+kc}{None}\PY{p}{,} \PY{n}{interval}\PY{o}{=}\PY{l+m+mi}{600}\PY{p}{)}
\PY{n}{ani\PYZus{}avr\PYZus{}fitness}\PY{o}{.}\PY{n}{save}\PY{p}{(}\PY{l+s+s1}{\PYZsq{}}\PY{l+s+s1}{animation\PYZus{}avr\PYZus{}fitness\PYZus{}sphere.gif}\PY{l+s+s1}{\PYZsq{}}\PY{p}{,} \PY{n}{writer}\PY{o}{=}\PY{l+s+s1}{\PYZsq{}}\PY{l+s+s1}{Pillow}\PY{l+s+s1}{\PYZsq{}}\PY{p}{,} \PY{n}{dpi}\PY{o}{=}\PY{l+m+mi}{100}\PY{p}{)}
\end{Verbatim}
\end{tcolorbox}

    \begin{Verbatim}[commandchars=\\\{\}]
MovieWriter Pillow unavailable; using Pillow instead.
    \end{Verbatim}

    \begin{center}
    \adjustimage{max size={0.9\linewidth}{0.9\paperheight}}{algorytmy_rojowe_files/algorytmy_rojowe_28_1.png}
    \end{center}
    { \hspace*{\fill} \\}
    
    \hypertarget{wykres-ux15bredniej-wartoux15bci-funkcji-celu-dla-kaux17cdej-czux105steczki-dla-funkcji-rastrigin}{%
\subsection{2.13 Wykres średniej wartości funkcji celu dla każdej
cząsteczki dla funkcji
rastrigin}\label{wykres-ux15bredniej-wartoux15bci-funkcji-celu-dla-kaux17cdej-czux105steczki-dla-funkcji-rastrigin}}

    \begin{tcolorbox}[breakable, size=fbox, boxrule=1pt, pad at break*=1mm,colback=cellbackground, colframe=cellborder]
\prompt{In}{incolor}{ }{\boxspacing}
\begin{Verbatim}[commandchars=\\\{\}]
\PY{c+c1}{\PYZsh{} Inicjalizacja wykresu dla funkcji sphere}
\PY{n}{fig\PYZus{}avr\PYZus{}fitness\PYZus{}r} \PY{o}{=} \PY{n}{plt}\PY{o}{.}\PY{n}{figure}\PY{p}{(}\PY{n}{figsize}\PY{o}{=}\PY{p}{(}\PY{l+m+mi}{15}\PY{p}{,} \PY{l+m+mi}{6}\PY{p}{)}\PY{p}{)}
\PY{n}{ax\PYZus{}avr\PYZus{}fitness\PYZus{}r} \PY{o}{=} \PY{n}{fig\PYZus{}avr\PYZus{}fitness\PYZus{}r}\PY{o}{.}\PY{n}{add\PYZus{}subplot}\PY{p}{(}\PY{l+m+mi}{111}\PY{p}{)}

\PY{n}{itteration} \PY{o}{=} \PY{p}{[}\PY{n}{itt} \PY{k}{for} \PY{n}{itt} \PY{o+ow}{in} \PY{n+nb}{range}\PY{p}{(}\PY{l+m+mi}{0}\PY{p}{,} \PY{n+nb}{len}\PY{p}{(}\PY{n}{avr\PYZus{}fitness\PYZus{}value\PYZus{}r}\PY{p}{)}\PY{p}{)}\PY{p}{]}

\PY{c+c1}{\PYZsh{} print(avr\PYZus{}fitness\PYZus{}value)}

\PY{k}{def} \PY{n+nf}{animate\PYZus{}avr\PYZus{}fitness\PYZus{}r}\PY{p}{(}\PY{n}{i}\PY{p}{)}\PY{p}{:}
    \PY{n}{ax\PYZus{}avr\PYZus{}fitness\PYZus{}r}\PY{o}{.}\PY{n}{clear}\PY{p}{(}\PY{p}{)}
    \PY{n}{ax\PYZus{}avr\PYZus{}fitness\PYZus{}r}\PY{o}{.}\PY{n}{set\PYZus{}title}\PY{p}{(}\PY{l+s+s1}{\PYZsq{}}\PY{l+s+s1}{Wykres wartości średniej funkcji fitness rastrigin względem każdej iteracji}\PY{l+s+s1}{\PYZsq{}}\PY{p}{)}
    \PY{n}{ax\PYZus{}avr\PYZus{}fitness\PYZus{}r}\PY{o}{.}\PY{n}{set\PYZus{}xticks}\PY{p}{(}\PY{n}{np}\PY{o}{.}\PY{n}{arange}\PY{p}{(}\PY{l+m+mi}{0}\PY{p}{,} \PY{n+nb}{len}\PY{p}{(}\PY{n}{avr\PYZus{}fitness\PYZus{}value\PYZus{}r}\PY{p}{)}\PY{p}{,} \PY{l+m+mi}{2}\PY{p}{)}\PY{p}{)}
    \PY{n}{ax\PYZus{}avr\PYZus{}fitness\PYZus{}r}\PY{o}{.}\PY{n}{set\PYZus{}xlim}\PY{p}{(}\PY{l+m+mi}{0}\PY{p}{,} \PY{l+m+mi}{100}\PY{p}{)}
    \PY{n}{ax\PYZus{}avr\PYZus{}fitness\PYZus{}r}\PY{o}{.}\PY{n}{set\PYZus{}ylim}\PY{p}{(}\PY{l+m+mi}{0}\PY{p}{,} \PY{l+m+mi}{1}\PY{p}{)}
    \PY{n}{ax\PYZus{}avr\PYZus{}fitness\PYZus{}r}\PY{o}{.}\PY{n}{set\PYZus{}xlabel}\PY{p}{(}\PY{l+s+s1}{\PYZsq{}}\PY{l+s+s1}{Numer iteracji}\PY{l+s+s1}{\PYZsq{}}\PY{p}{)}
    \PY{n}{ax\PYZus{}avr\PYZus{}fitness\PYZus{}r}\PY{o}{.}\PY{n}{set\PYZus{}ylabel}\PY{p}{(}\PY{l+s+s1}{\PYZsq{}}\PY{l+s+s1}{Średnia wartość funkcji}\PY{l+s+s1}{\PYZsq{}}\PY{p}{)}
    \PY{n}{data} \PY{o}{=} \PY{n+nb}{list}\PY{p}{(}\PY{n+nb}{zip}\PY{p}{(}\PY{n}{itteration}\PY{p}{,} \PY{n}{avr\PYZus{}fitness\PYZus{}value\PYZus{}r}\PY{p}{[}\PY{l+m+mi}{0}\PY{p}{:}\PY{n}{i}\PY{o}{+}\PY{l+m+mi}{1}\PY{p}{]}\PY{p}{)}\PY{p}{)}
    \PY{n}{ax\PYZus{}avr\PYZus{}fitness\PYZus{}r}\PY{o}{.}\PY{n}{plot}\PY{p}{(}\PY{o}{*}\PY{n+nb}{zip}\PY{p}{(}\PY{o}{*}\PY{n}{data}\PY{p}{)}\PY{p}{,} \PY{n}{color}\PY{o}{=}\PY{l+s+s1}{\PYZsq{}}\PY{l+s+s1}{r}\PY{l+s+s1}{\PYZsq{}}\PY{p}{)}

\PY{n}{ani\PYZus{}avr\PYZus{}fitness\PYZus{}r} \PY{o}{=} \PY{n}{FuncAnimation}\PY{p}{(}\PY{n}{fig\PYZus{}avr\PYZus{}fitness\PYZus{}r}\PY{p}{,} \PY{n}{animate\PYZus{}avr\PYZus{}fitness\PYZus{}r}\PY{p}{,} \PY{n}{frames}\PY{o}{=}\PY{n+nb}{len}\PY{p}{(}\PY{n}{avr\PYZus{}fitness\PYZus{}value\PYZus{}r}\PY{p}{)}\PY{p}{,} \PY{n}{init\PYZus{}func}\PY{o}{=}\PY{k}{lambda}\PY{p}{:} \PY{k+kc}{None}\PY{p}{,} \PY{n}{interval}\PY{o}{=}\PY{l+m+mi}{600}\PY{p}{)}
\PY{n}{ani\PYZus{}avr\PYZus{}fitness\PYZus{}r}\PY{o}{.}\PY{n}{save}\PY{p}{(}\PY{l+s+s1}{\PYZsq{}}\PY{l+s+s1}{animation\PYZus{}avr\PYZus{}fitness\PYZus{}rastrigin.gif}\PY{l+s+s1}{\PYZsq{}}\PY{p}{,} \PY{n}{writer}\PY{o}{=}\PY{l+s+s1}{\PYZsq{}}\PY{l+s+s1}{Pillow}\PY{l+s+s1}{\PYZsq{}}\PY{p}{,} \PY{n}{dpi}\PY{o}{=}\PY{l+m+mi}{100}\PY{p}{)}
\end{Verbatim}
\end{tcolorbox}

    \begin{Verbatim}[commandchars=\\\{\}]
MovieWriter Pillow unavailable; using Pillow instead.
    \end{Verbatim}

    \begin{center}
    \adjustimage{max size={0.9\linewidth}{0.9\paperheight}}{algorytmy_rojowe_files/algorytmy_rojowe_30_1.png}
    \end{center}
    { \hspace*{\fill} \\}
    
    \hypertarget{algorytm-mruxf3wkowy}{%
\section{Algorytm mrówkowy}\label{algorytm-mruxf3wkowy}}

    \hypertarget{teoria}{%
\subsection{3 Teoria}\label{teoria}}

\hypertarget{wstux119p}{%
\subsubsection{3.1 Wstęp}\label{wstux119p}}

Algorytm mrówkowy, znany również jako optymalizacja mrówkowa (ang. ant
colony optimization, ACO), jest metaheurystyczną metodą rozwiązywania
problemów optymalizacyjnych, która naśladuje zachowanie kolonii mrówek w
poszukiwaniu najkrótszej trasy do źródła pokarmu.

Algorytm mrówkowy opiera się na działaniu wielu wirtualnych mrówek,
które poruszają się po grafie reprezentującym problem optymalizacyjny.
Każda mrówka wybiera kolejne kroki na podstawie informacji lokalnych i
globalnych. Informacje lokalne to feromony pozostawiane przez inne
mrówki na odwiedzanych ścieżkach, a globalne to heurystyka, która
określa atrakcyjność danego kierunku na podstawie pewnych heurystycznych
informacji.

W początkowej fazie algorytmu mrówki poruszają się losowo po grafie, a w
miarę upływu czasu wybierają coraz bardziej optymalne trasy na podstawie
informacji pozostawianych przez siebie i inne mrówki. Feromony są
aktualizowane na podstawie jakości wybranych rozwiązań, co prowadzi do
wzmacniania atrakcyjności lepszych ścieżek.

Algorytm mrówkowy znalazł zastosowanie w wielu dziedzinach, takich jak
problem komiwojażera, układanie planów, routing w sieciach
telekomunikacyjnych, projektowanie układów elektronicznych i wiele
innych problemów optymalizacyjnych. Dzięki swojej zdolności do
znajdowania zbliżonych do optymalnych rozwiązań w czasie rzeczywistym,
algorytmy mrówkowe są często wykorzystywane w sytuacjach, w których
tradycyjne metody optymalizacyjne mogą być nieefektywne lub
niewystarczające.

\hypertarget{schemat-dziaux142ania-algorytmu-mruxf3wkowego}{%
\subsubsection{3.2 Schemat działania algorytmu
mrówkowego}\label{schemat-dziaux142ania-algorytmu-mruxf3wkowego}}

\begin{enumerate}
\def\labelenumi{\arabic{enumi}.}
\tightlist
\item
  Inicjalizacja:

  \begin{itemize}
  \tightlist
  \item
    Tworzenie grafu lub planszy, na której odbywać się będzie
    poszukiwanie rozwiązania.
  \item
    Losowe rozmieszczenie mrówek w początkowych punktach.
  \end{itemize}
\item
  Przesuwanie się mrówek:

  \begin{itemize}
  \tightlist
  \item
    Każda mrówka porusza się po grafie/planszy, odwiedzając kolejne
    wierzchołki/komórki.
  \item
    Przy wyborze kolejnego ruchu, mrówka kieruje się informacjami
    lokalnymi i globalnymi.
  \end{itemize}
\item
  Informacje lokalne:

  \begin{itemize}
  \tightlist
  \item
    Mrówka odczuwa informacje lokalne, które mogą obejmować:

    \begin{itemize}
    \tightlist
    \item
      Feromony pozostawione przez inne mrówki na odwiedzonych ścieżkach.
    \item
      Heurystyki opisujące atrakcyjność danego ruchu, np. odległość do
      celu.
    \end{itemize}
  \end{itemize}
\item
  Informacje globalne:

  \begin{itemize}
  \tightlist
  \item
    Mrówka uwzględnia informacje globalne, które mogą zawierać:

    \begin{itemize}
    \tightlist
    \item
      Ogólne właściwości grafu/planszy, np. odległości między
      wierzchołkami.
    \item
      Ogólne kryteria optymalizacyjne.
    \end{itemize}
  \end{itemize}
\item
  Aktualizacja feromonów:

  \begin{itemize}
  \tightlist
  \item
    Gdy mrówka przechodzi przez krawędź, pozostawia tam feromony.
  \item
    Feromony na krawędziach są stopniowo aktualizowane, biorąc pod uwagę
    jakość rozwiązania.
  \end{itemize}
\item
  Powtarzanie procesu:

  \begin{itemize}
  \tightlist
  \item
    Proces przesuwania się mrówek i aktualizacji feromonów jest
    powtarzany przez określoną liczbę iteracji.
  \item
    Może być również zdefiniowany warunek zakończenia, np. czas trwania,
    brak znaczących zmian.
  \end{itemize}
\item
  Wybór rozwiązania:

  \begin{itemize}
  \tightlist
  \item
    Po zakończeniu iteracji, można wybrać najlepsze znalezione
    rozwiązanie.
  \item
    Najlepsze rozwiązanie może być wybrane na podstawie ilości feromonów
    na odwiedzonych krawędziach lub innych kryteriów.
  \end{itemize}
\end{enumerate}

Obliczanie prawdopodobieństwa przejście z wierzchołka i do wierzchołka j
przez mrówkę k

Pijk=(ij)\emph{(ij)lJik(il)}(il)

gdzie: - Pijk to prawdopodobieństwo przejścia z wierzchołka i do
wierzchołka j - ij to stężenie feromonu na krawędzi między wierzchołkami
i i j - ij to atrakcyjność przejścia z wierzchołka i do j - alfa i beta
to parametry, które kontrolują wpływ intensywności feromonu i
atrakcyjności przejścia - Jik to zbiór wierzchołków dostępnych dla
mrówki k z wierzchołka i

Aktualizacja fermonu na ścieżce:

ij=(1-)ij +ijk

Gdzie: - ij to intensywność feromonu na ścieżce między i-tym a j-tym
wierzchołkiem. ˙ - ijk to ilość feromonu, którą k-ta mrówka zdeponowała
na ścieżce między i-tym a j-tym wierzchołkiem. - to współczynnik
parowania feromonu, który jest zazwyczaj ustalany na wartość pomiędzy 0
a 1.

Obliczanie ijk dla każdej mrówki po przejściu ścieżki

ijk=QLk

Gdzie: - Q to parametr określający ilość fermonów pozostawionych przez
mrówkę na ścieżce - Lk to długość ścieżki przebytej przez mrówkę k

    \hypertarget{rozwiux105zanie}{%
\subsection{4. Rozwiązanie}\label{rozwiux105zanie}}

    \hypertarget{importowanie-potrzebnych-bibliotek}{%
\subsection{4.1 Importowanie potrzebnych
bibliotek}\label{importowanie-potrzebnych-bibliotek}}

    \begin{tcolorbox}[breakable, size=fbox, boxrule=1pt, pad at break*=1mm,colback=cellbackground, colframe=cellborder]
\prompt{In}{incolor}{ }{\boxspacing}
\begin{Verbatim}[commandchars=\\\{\}]
\PY{k+kn}{import} \PY{n+nn}{numpy} \PY{k}{as} \PY{n+nn}{np}
\PY{k+kn}{import} \PY{n+nn}{matplotlib}\PY{n+nn}{.}\PY{n+nn}{pyplot} \PY{k}{as} \PY{n+nn}{plt}
\PY{k+kn}{from} \PY{n+nn}{matplotlib}\PY{n+nn}{.}\PY{n+nn}{animation} \PY{k+kn}{import} \PY{n}{FuncAnimation}
\PY{k+kn}{from} \PY{n+nn}{mpl\PYZus{}toolkits}\PY{n+nn}{.}\PY{n+nn}{mplot3d} \PY{k+kn}{import} \PY{n}{Axes3D}
\end{Verbatim}
\end{tcolorbox}

    \hypertarget{zdefiniowanie-funkcji-celu}{%
\subsection{4.2 Zdefiniowanie funkcji
celu}\label{zdefiniowanie-funkcji-celu}}

Funkcja ``sphere\_function'' oblicza wartość funkcji celu dla podanych
wartości x i y. Funkcja ta jest reprezentacją funkcji sfery, która jest
zdefiniowana jako suma kwadratów wartości x i y. Zwraca wynik tej sumy.

Funkcja ``rastrigin\_function'' również oblicza wartość funkcji celu dla
podanych wartości x i y. Ta funkcja reprezentuje funkcję Rastrigina,
która jest zdefiniowana jako suma kilku składników. Pierwszy składnik to
20, a pozostałe składniki zawierają kwadraty wartości x i y, a także
obliczenia kosinusów na podstawie tych wartości. Funkcja zwraca wynik
sumy tych składników.

    \begin{tcolorbox}[breakable, size=fbox, boxrule=1pt, pad at break*=1mm,colback=cellbackground, colframe=cellborder]
\prompt{In}{incolor}{ }{\boxspacing}
\begin{Verbatim}[commandchars=\\\{\}]
\PY{k}{def} \PY{n+nf}{sphere\PYZus{}function}\PY{p}{(}\PY{n}{x}\PY{p}{,} \PY{n}{y}\PY{p}{)}\PY{p}{:}
    \PY{l+s+sd}{\PYZdq{}\PYZdq{}\PYZdq{}Funkcja celu \PYZhy{} sfera\PYZdq{}\PYZdq{}\PYZdq{}}
    \PY{k}{return} \PY{n}{x}\PY{o}{*}\PY{o}{*}\PY{l+m+mi}{2} \PY{o}{+} \PY{n}{y}\PY{o}{*}\PY{o}{*}\PY{l+m+mi}{2}

\PY{k}{def} \PY{n+nf}{rastrigin\PYZus{}function}\PY{p}{(}\PY{n}{x}\PY{p}{,} \PY{n}{y}\PY{p}{)}\PY{p}{:}
    \PY{l+s+sd}{\PYZdq{}\PYZdq{}\PYZdq{}Funkcja celu \PYZhy{} Rastrigin\PYZdq{}\PYZdq{}\PYZdq{}}
    \PY{k}{return} \PY{l+m+mi}{20} \PY{o}{+} \PY{n}{x}\PY{o}{*}\PY{o}{*}\PY{l+m+mi}{2} \PY{o}{\PYZhy{}} \PY{l+m+mi}{10} \PY{o}{*} \PY{n}{np}\PY{o}{.}\PY{n}{cos}\PY{p}{(}\PY{l+m+mi}{2} \PY{o}{*} \PY{n}{np}\PY{o}{.}\PY{n}{pi} \PY{o}{*} \PY{n}{x}\PY{p}{)} \PY{o}{+} \PY{n}{y}\PY{o}{*}\PY{o}{*}\PY{l+m+mi}{2} \PY{o}{\PYZhy{}} \PY{l+m+mi}{10} \PY{o}{*} \PY{n}{np}\PY{o}{.}\PY{n}{cos}\PY{p}{(}\PY{l+m+mi}{2} \PY{o}{*} \PY{n}{np}\PY{o}{.}\PY{n}{pi} \PY{o}{*} \PY{n}{y}\PY{p}{)}
\end{Verbatim}
\end{tcolorbox}

    \hypertarget{ustalamy-parametry-optymalizacji}{%
\subsection{4.3 Ustalamy parametry
optymalizacji}\label{ustalamy-parametry-optymalizacji}}

Klasa Ant reprezentuje mrówkę. Konstruktor klasy inicjalizuje obiekt Ant
z początkowymi współrzędnymi x i y. Dodatkowo, oblicza wartość funkcji
sfery dla tych współrzędnych i przechowuje ją w zmiennej z.

    \begin{tcolorbox}[breakable, size=fbox, boxrule=1pt, pad at break*=1mm,colback=cellbackground, colframe=cellborder]
\prompt{In}{incolor}{ }{\boxspacing}
\begin{Verbatim}[commandchars=\\\{\}]
\PY{k}{class} \PY{n+nc}{Ant}\PY{p}{:}
    \PY{k}{def} \PY{n+nf+fm}{\PYZus{}\PYZus{}init\PYZus{}\PYZus{}}\PY{p}{(}\PY{n+nb+bp}{self}\PY{p}{,} \PY{n}{x}\PY{p}{,} \PY{n}{y}\PY{p}{)}\PY{p}{:}
        \PY{n+nb+bp}{self}\PY{o}{.}\PY{n}{x} \PY{o}{=} \PY{n}{x}
        \PY{n+nb+bp}{self}\PY{o}{.}\PY{n}{y} \PY{o}{=} \PY{n}{y}
        \PY{n+nb+bp}{self}\PY{o}{.}\PY{n}{z} \PY{o}{=} \PY{n}{sphere\PYZus{}function}\PY{p}{(}\PY{n}{x}\PY{p}{,} \PY{n}{y}\PY{p}{)}
\end{Verbatim}
\end{tcolorbox}

    \hypertarget{funkcja-update}{%
\subsection{4.4 Funkcja Update}\label{funkcja-update}}

Funkcja update\_ant aktualizuje współrzędne mrówki ant. Tworzy nowe
wartości new\_x i new\_y, które są zaktualizowanymi współrzędnymi x i y
mrówki, dodając do nich losową wartość z zakresu {[}-step\_size,
step\_size{]}. Następnie oblicza nową wartość funkcji sfery dla
zaktualizowanych współrzędnych (new\_x, new\_y) i porównuje ją z
poprzednią wartością z mrówki (ant.z). Jeśli nowa wartość jest mniejsza,
to aktualizuje współrzędne x, y i z mrówki.

    \begin{tcolorbox}[breakable, size=fbox, boxrule=1pt, pad at break*=1mm,colback=cellbackground, colframe=cellborder]
\prompt{In}{incolor}{ }{\boxspacing}
\begin{Verbatim}[commandchars=\\\{\}]
\PY{k}{def} \PY{n+nf}{update\PYZus{}ant}\PY{p}{(}\PY{n}{ant}\PY{p}{,} \PY{n}{step\PYZus{}size}\PY{p}{)}\PY{p}{:}
    \PY{n}{delta\PYZus{}x} \PY{o}{=} \PY{n}{np}\PY{o}{.}\PY{n}{random}\PY{o}{.}\PY{n}{uniform}\PY{p}{(}\PY{o}{\PYZhy{}}\PY{n}{step\PYZus{}size}\PY{p}{,} \PY{n}{step\PYZus{}size}\PY{p}{)}
    \PY{n}{delta\PYZus{}y} \PY{o}{=} \PY{n}{np}\PY{o}{.}\PY{n}{random}\PY{o}{.}\PY{n}{uniform}\PY{p}{(}\PY{o}{\PYZhy{}}\PY{n}{step\PYZus{}size}\PY{p}{,} \PY{n}{step\PYZus{}size}\PY{p}{)}
    \PY{n}{new\PYZus{}x} \PY{o}{=} \PY{n}{ant}\PY{o}{.}\PY{n}{x} \PY{o}{+} \PY{n}{delta\PYZus{}x}
    \PY{n}{new\PYZus{}y} \PY{o}{=} \PY{n}{ant}\PY{o}{.}\PY{n}{y} \PY{o}{+} \PY{n}{delta\PYZus{}y}
    \PY{n}{new\PYZus{}z} \PY{o}{=} \PY{n}{rastrigin\PYZus{}function}\PY{p}{(}\PY{n}{new\PYZus{}x}\PY{p}{,} \PY{n}{new\PYZus{}y}\PY{p}{)}


    \PY{k}{if} \PY{n}{new\PYZus{}z} \PY{o}{\PYZlt{}} \PY{n}{ant}\PY{o}{.}\PY{n}{z}\PY{p}{:}
        \PY{n}{ant}\PY{o}{.}\PY{n}{x} \PY{o}{=} \PY{n}{new\PYZus{}x}
        \PY{n}{ant}\PY{o}{.}\PY{n}{y} \PY{o}{=} \PY{n}{new\PYZus{}y}
        \PY{n}{ant}\PY{o}{.}\PY{n}{z} \PY{o}{=} \PY{n}{new\PYZus{}z}
\end{Verbatim}
\end{tcolorbox}

    \hypertarget{wykresy-dla-funkcji-celu---sfera}{%
\subsection{4.5 Wykresy dla Funkcji celu -
sfera}\label{wykresy-dla-funkcji-celu---sfera}}

Funkcja plot\_animation tworzy animację wizualizując działanie algorytmu
mrówkowego dla funkcji sfery. Tworzone są dwa wykresy: jeden w trzech
wymiarach (fig\_3d) i drugi w dwóch wymiarach (fig\_2d).

Na początku kodu tworzona jest populacja mrówek (ants) i następnie
uruchamiana jest animacja dla określonej liczby kroków (num\_steps) i
rozmiaru kroku (step\_size).

W trzech wymiarach, najpierw tworzona jest siatka X i Y dla zakresów
x\_range i y\_range. Następnie obliczane są wartości Z funkcji sfery dla
każdego punktu siatki. Wykres trójwymiarowy przedstawia powierzchnię
funkcji sfery, a także wyświetla punkty reprezentujące współrzędne
mrówek (ant\_points) oraz punkt najlepszego wyniku (best\_point).
Dodatkowo, na wykresie wypisywane są informacje o iteracji i najlepszym
wyniku.

W dwóch wymiarach, tworzony jest wykres konturowy funkcji sfery, a także
wyświetlane są punkty mrówek i punkt najlepszego wyniku. Również
wypisywane są informacje o iteracji i najlepszym wyniku.

Funkcja update(frame) jest funkcją wywoływaną w każdej iteracji
animacji. Aktualizuje współrzędne mrówek, wykresy i teksty informacyjne.

Na koniec, tworzone są animacje (anim i anim\_2d) i zapisywane do plików
``sphere\_3d.gif'' i ``sphere.gif''. Następnie animacje są wyświetlane
przy użyciu plt.show().

    \begin{tcolorbox}[breakable, size=fbox, boxrule=1pt, pad at break*=1mm,colback=cellbackground, colframe=cellborder]
\prompt{In}{incolor}{ }{\boxspacing}
\begin{Verbatim}[commandchars=\\\{\}]
\PY{k}{def} \PY{n+nf}{plot\PYZus{}animation}\PY{p}{(}\PY{n}{ants}\PY{p}{,} \PY{n}{num\PYZus{}steps}\PY{p}{,} \PY{n}{step\PYZus{}size}\PY{p}{)}\PY{p}{:}
    \PY{n}{fig\PYZus{}3d} \PY{o}{=} \PY{n}{plt}\PY{o}{.}\PY{n}{figure}\PY{p}{(}\PY{p}{)}
    \PY{n}{ax\PYZus{}3d} \PY{o}{=} \PY{n}{fig\PYZus{}3d}\PY{o}{.}\PY{n}{add\PYZus{}subplot}\PY{p}{(}\PY{l+m+mi}{111}\PY{p}{,} \PY{n}{projection}\PY{o}{=}\PY{l+s+s1}{\PYZsq{}}\PY{l+s+s1}{3d}\PY{l+s+s1}{\PYZsq{}}\PY{p}{)}
   
    \PY{n}{fig\PYZus{}2d} \PY{o}{=} \PY{n}{plt}\PY{o}{.}\PY{n}{figure}\PY{p}{(}\PY{p}{)}
    \PY{n}{ax\PYZus{}2d} \PY{o}{=} \PY{n}{fig\PYZus{}2d}\PY{o}{.}\PY{n}{add\PYZus{}subplot}\PY{p}{(}\PY{l+m+mi}{111}\PY{p}{)}
   
    \PY{n}{x\PYZus{}range} \PY{o}{=} \PY{n}{np}\PY{o}{.}\PY{n}{linspace}\PY{p}{(}\PY{o}{\PYZhy{}}\PY{l+m+mi}{10}\PY{p}{,} \PY{l+m+mi}{10}\PY{p}{,} \PY{l+m+mi}{200}\PY{p}{)}  \PY{c+c1}{\PYZsh{} Zwiększenie liczby próbek x}
    \PY{n}{y\PYZus{}range} \PY{o}{=} \PY{n}{np}\PY{o}{.}\PY{n}{linspace}\PY{p}{(}\PY{o}{\PYZhy{}}\PY{l+m+mi}{10}\PY{p}{,} \PY{l+m+mi}{10}\PY{p}{,} \PY{l+m+mi}{200}\PY{p}{)}  \PY{c+c1}{\PYZsh{} Zwiększenie liczby próbek y}
    \PY{n}{X}\PY{p}{,} \PY{n}{Y} \PY{o}{=} \PY{n}{np}\PY{o}{.}\PY{n}{meshgrid}\PY{p}{(}\PY{n}{x\PYZus{}range}\PY{p}{,} \PY{n}{y\PYZus{}range}\PY{p}{)}
    \PY{n}{Z} \PY{o}{=} \PY{n}{sphere\PYZus{}function}\PY{p}{(}\PY{n}{X}\PY{p}{,} \PY{n}{Y}\PY{p}{)}
   
    \PY{n}{ax\PYZus{}3d}\PY{o}{.}\PY{n}{plot\PYZus{}surface}\PY{p}{(}\PY{n}{X}\PY{p}{,} \PY{n}{Y}\PY{p}{,} \PY{n}{Z}\PY{p}{,} \PY{n}{cmap}\PY{o}{=}\PY{l+s+s1}{\PYZsq{}}\PY{l+s+s1}{viridis}\PY{l+s+s1}{\PYZsq{}}\PY{p}{,} \PY{n}{alpha}\PY{o}{=}\PY{l+m+mf}{0.8}\PY{p}{)}
   
    \PY{n}{ant\PYZus{}points} \PY{o}{=} \PY{n}{ax\PYZus{}3d}\PY{o}{.}\PY{n}{scatter}\PY{p}{(}\PY{p}{[}\PY{n}{ant}\PY{o}{.}\PY{n}{x} \PY{k}{for} \PY{n}{ant} \PY{o+ow}{in} \PY{n}{ants}\PY{p}{]}\PY{p}{,} \PY{p}{[}\PY{n}{ant}\PY{o}{.}\PY{n}{y} \PY{k}{for} \PY{n}{ant} \PY{o+ow}{in} \PY{n}{ants}\PY{p}{]}\PY{p}{,} \PY{p}{[}\PY{n}{ant}\PY{o}{.}\PY{n}{z} \PY{k}{for} \PY{n}{ant} \PY{o+ow}{in} \PY{n}{ants}\PY{p}{]}\PY{p}{,} \PY{n}{color}\PY{o}{=}\PY{l+s+s1}{\PYZsq{}}\PY{l+s+s1}{blue}\PY{l+s+s1}{\PYZsq{}}\PY{p}{)}
    \PY{n}{best\PYZus{}ant} \PY{o}{=} \PY{n+nb}{min}\PY{p}{(}\PY{n}{ants}\PY{p}{,} \PY{n}{key}\PY{o}{=}\PY{k}{lambda} \PY{n}{ant}\PY{p}{:} \PY{n}{ant}\PY{o}{.}\PY{n}{z}\PY{p}{)}
    \PY{n}{best\PYZus{}point} \PY{o}{=} \PY{n}{ax\PYZus{}3d}\PY{o}{.}\PY{n}{scatter}\PY{p}{(}\PY{n}{best\PYZus{}ant}\PY{o}{.}\PY{n}{x}\PY{p}{,} \PY{n}{best\PYZus{}ant}\PY{o}{.}\PY{n}{y}\PY{p}{,} \PY{n}{best\PYZus{}ant}\PY{o}{.}\PY{n}{z}\PY{p}{,} \PY{n}{color}\PY{o}{=}\PY{l+s+s1}{\PYZsq{}}\PY{l+s+s1}{red}\PY{l+s+s1}{\PYZsq{}}\PY{p}{)}
    \PY{n}{iter\PYZus{}text\PYZus{}3d} \PY{o}{=} \PY{n}{ax\PYZus{}3d}\PY{o}{.}\PY{n}{text2D}\PY{p}{(}\PY{l+m+mf}{0.05}\PY{p}{,} \PY{l+m+mf}{0.95}\PY{p}{,} \PY{l+s+s2}{\PYZdq{}}\PY{l+s+s2}{\PYZdq{}}\PY{p}{,} \PY{n}{transform}\PY{o}{=}\PY{n}{ax\PYZus{}3d}\PY{o}{.}\PY{n}{transAxes}\PY{p}{)}
    \PY{n}{best\PYZus{}text\PYZus{}3d} \PY{o}{=} \PY{n}{ax\PYZus{}3d}\PY{o}{.}\PY{n}{text2D}\PY{p}{(}\PY{l+m+mf}{0.05}\PY{p}{,} \PY{l+m+mf}{0.90}\PY{p}{,} \PY{l+s+s2}{\PYZdq{}}\PY{l+s+s2}{\PYZdq{}}\PY{p}{,} \PY{n}{transform}\PY{o}{=}\PY{n}{ax\PYZus{}3d}\PY{o}{.}\PY{n}{transAxes}\PY{p}{)}
   
    \PY{n}{ax\PYZus{}3d}\PY{o}{.}\PY{n}{set\PYZus{}xlabel}\PY{p}{(}\PY{l+s+s1}{\PYZsq{}}\PY{l+s+s1}{X}\PY{l+s+s1}{\PYZsq{}}\PY{p}{)}
    \PY{n}{ax\PYZus{}3d}\PY{o}{.}\PY{n}{set\PYZus{}ylabel}\PY{p}{(}\PY{l+s+s1}{\PYZsq{}}\PY{l+s+s1}{Y}\PY{l+s+s1}{\PYZsq{}}\PY{p}{)}
    \PY{n}{ax\PYZus{}3d}\PY{o}{.}\PY{n}{set\PYZus{}zlabel}\PY{p}{(}\PY{l+s+s1}{\PYZsq{}}\PY{l+s+s1}{Z}\PY{l+s+s1}{\PYZsq{}}\PY{p}{)}
    \PY{n}{ax\PYZus{}3d}\PY{o}{.}\PY{n}{set\PYZus{}title}\PY{p}{(}\PY{l+s+s1}{\PYZsq{}}\PY{l+s+s1}{Algorytm mrówkowy \PYZhy{} funkcja sfera}\PY{l+s+s1}{\PYZsq{}}\PY{p}{)}
   
    \PY{n}{iter\PYZus{}text\PYZus{}2d} \PY{o}{=} \PY{n}{ax\PYZus{}2d}\PY{o}{.}\PY{n}{text}\PY{p}{(}\PY{l+m+mf}{0.05}\PY{p}{,} \PY{l+m+mf}{0.95}\PY{p}{,} \PY{l+s+s2}{\PYZdq{}}\PY{l+s+s2}{\PYZdq{}}\PY{p}{,} \PY{n}{transform}\PY{o}{=}\PY{n}{ax\PYZus{}2d}\PY{o}{.}\PY{n}{transAxes}\PY{p}{)}
    \PY{n}{best\PYZus{}text\PYZus{}2d} \PY{o}{=} \PY{n}{ax\PYZus{}2d}\PY{o}{.}\PY{n}{text}\PY{p}{(}\PY{l+m+mf}{0.05}\PY{p}{,} \PY{l+m+mf}{0.90}\PY{p}{,} \PY{l+s+s2}{\PYZdq{}}\PY{l+s+s2}{\PYZdq{}}\PY{p}{,} \PY{n}{transform}\PY{o}{=}\PY{n}{ax\PYZus{}2d}\PY{o}{.}\PY{n}{transAxes}\PY{p}{)}
   
    \PY{k}{def} \PY{n+nf}{update}\PY{p}{(}\PY{n}{frame}\PY{p}{)}\PY{p}{:}
        \PY{k}{for} \PY{n}{ant} \PY{o+ow}{in} \PY{n}{ants}\PY{p}{:}
            \PY{n}{update\PYZus{}ant}\PY{p}{(}\PY{n}{ant}\PY{p}{,} \PY{n}{step\PYZus{}size}\PY{p}{)}
       
        \PY{n}{ant\PYZus{}points}\PY{o}{.}\PY{n}{\PYZus{}offsets3d} \PY{o}{=} \PY{p}{(}\PY{p}{[}\PY{n}{ant}\PY{o}{.}\PY{n}{x} \PY{k}{for} \PY{n}{ant} \PY{o+ow}{in} \PY{n}{ants}\PY{p}{]}\PY{p}{,} \PY{p}{[}\PY{n}{ant}\PY{o}{.}\PY{n}{y} \PY{k}{for} \PY{n}{ant} \PY{o+ow}{in} \PY{n}{ants}\PY{p}{]}\PY{p}{,} \PY{p}{[}\PY{n}{ant}\PY{o}{.}\PY{n}{z} \PY{k}{for} \PY{n}{ant} \PY{o+ow}{in} \PY{n}{ants}\PY{p}{]}\PY{p}{)}
        \PY{n}{best\PYZus{}ant} \PY{o}{=} \PY{n+nb}{min}\PY{p}{(}\PY{n}{ants}\PY{p}{,} \PY{n}{key}\PY{o}{=}\PY{k}{lambda} \PY{n}{ant}\PY{p}{:} \PY{n}{ant}\PY{o}{.}\PY{n}{z}\PY{p}{)}
        \PY{n}{best\PYZus{}point}\PY{o}{.}\PY{n}{\PYZus{}offsets3d} \PY{o}{=} \PY{p}{(}\PY{p}{[}\PY{n}{best\PYZus{}ant}\PY{o}{.}\PY{n}{x}\PY{p}{]}\PY{p}{,} \PY{p}{[}\PY{n}{best\PYZus{}ant}\PY{o}{.}\PY{n}{y}\PY{p}{]}\PY{p}{,} \PY{p}{[}\PY{n}{best\PYZus{}ant}\PY{o}{.}\PY{n}{z}\PY{p}{]}\PY{p}{)}
        \PY{n}{iter\PYZus{}text\PYZus{}3d}\PY{o}{.}\PY{n}{set\PYZus{}text}\PY{p}{(}\PY{l+s+sa}{f}\PY{l+s+s2}{\PYZdq{}}\PY{l+s+s2}{Iteracja: }\PY{l+s+si}{\PYZob{}}\PY{n}{frame}\PY{o}{+}\PY{l+m+mi}{1}\PY{l+s+si}{\PYZcb{}}\PY{l+s+s2}{\PYZdq{}}\PY{p}{)}
        \PY{n}{best\PYZus{}text\PYZus{}3d}\PY{o}{.}\PY{n}{set\PYZus{}text}\PY{p}{(}\PY{l+s+sa}{f}\PY{l+s+s2}{\PYZdq{}}\PY{l+s+s2}{Najlepszy wynik: }\PY{l+s+si}{\PYZob{}}\PY{n}{best\PYZus{}ant}\PY{o}{.}\PY{n}{z}\PY{l+s+si}{:}\PY{l+s+s2}{.7f}\PY{l+s+si}{\PYZcb{}}\PY{l+s+s2}{\PYZdq{}}\PY{p}{)}
       
        \PY{n}{ax\PYZus{}2d}\PY{o}{.}\PY{n}{clear}\PY{p}{(}\PY{p}{)}
        \PY{n}{ax\PYZus{}2d}\PY{o}{.}\PY{n}{contourf}\PY{p}{(}\PY{n}{X}\PY{p}{,} \PY{n}{Y}\PY{p}{,} \PY{n}{Z}\PY{p}{,} \PY{n}{cmap}\PY{o}{=}\PY{l+s+s1}{\PYZsq{}}\PY{l+s+s1}{viridis}\PY{l+s+s1}{\PYZsq{}}\PY{p}{,} \PY{n}{alpha}\PY{o}{=}\PY{l+m+mf}{0.8}\PY{p}{)}
        \PY{n}{ax\PYZus{}2d}\PY{o}{.}\PY{n}{plot}\PY{p}{(}\PY{p}{[}\PY{n}{ant}\PY{o}{.}\PY{n}{x} \PY{k}{for} \PY{n}{ant} \PY{o+ow}{in} \PY{n}{ants}\PY{p}{]}\PY{p}{,} \PY{p}{[}\PY{n}{ant}\PY{o}{.}\PY{n}{y} \PY{k}{for} \PY{n}{ant} \PY{o+ow}{in} \PY{n}{ants}\PY{p}{]}\PY{p}{,} \PY{l+s+s1}{\PYZsq{}}\PY{l+s+s1}{bo}\PY{l+s+s1}{\PYZsq{}}\PY{p}{)}
        \PY{n}{ax\PYZus{}2d}\PY{o}{.}\PY{n}{plot}\PY{p}{(}\PY{n}{best\PYZus{}ant}\PY{o}{.}\PY{n}{x}\PY{p}{,} \PY{n}{best\PYZus{}ant}\PY{o}{.}\PY{n}{y}\PY{p}{,} \PY{l+s+s1}{\PYZsq{}}\PY{l+s+s1}{ro}\PY{l+s+s1}{\PYZsq{}}\PY{p}{)}
        \PY{n}{iter\PYZus{}text\PYZus{}2d}\PY{o}{.}\PY{n}{set\PYZus{}text}\PY{p}{(}\PY{l+s+sa}{f}\PY{l+s+s2}{\PYZdq{}}\PY{l+s+s2}{Iteracja: }\PY{l+s+si}{\PYZob{}}\PY{n}{frame}\PY{o}{+}\PY{l+m+mi}{1}\PY{l+s+si}{\PYZcb{}}\PY{l+s+s2}{\PYZdq{}}\PY{p}{)}
        \PY{n}{best\PYZus{}text\PYZus{}2d}\PY{o}{.}\PY{n}{set\PYZus{}text}\PY{p}{(}\PY{l+s+sa}{f}\PY{l+s+s2}{\PYZdq{}}\PY{l+s+s2}{Najlepszy wynik: }\PY{l+s+si}{\PYZob{}}\PY{n}{best\PYZus{}ant}\PY{o}{.}\PY{n}{z}\PY{l+s+si}{:}\PY{l+s+s2}{.7f}\PY{l+s+si}{\PYZcb{}}\PY{l+s+s2}{\PYZdq{}}\PY{p}{)}
       
    \PY{n}{anim} \PY{o}{=} \PY{n}{animation}\PY{o}{.}\PY{n}{FuncAnimation}\PY{p}{(}\PY{n}{fig\PYZus{}3d}\PY{p}{,} \PY{n}{update}\PY{p}{,} \PY{n}{frames}\PY{o}{=}\PY{n}{num\PYZus{}steps}\PY{p}{,} \PY{n}{interval}\PY{o}{=}\PY{l+m+mi}{200}\PY{p}{,} \PY{n}{repeat}\PY{o}{=}\PY{k+kc}{False}\PY{p}{)}
    \PY{n}{anim\PYZus{}2d} \PY{o}{=} \PY{n}{animation}\PY{o}{.}\PY{n}{FuncAnimation}\PY{p}{(}\PY{n}{fig\PYZus{}2d}\PY{p}{,} \PY{n}{update}\PY{p}{,} \PY{n}{frames}\PY{o}{=}\PY{n}{num\PYZus{}steps}\PY{p}{,} \PY{n}{interval}\PY{o}{=}\PY{l+m+mi}{200}\PY{p}{,} \PY{n}{repeat}\PY{o}{=}\PY{k+kc}{False}\PY{p}{)}
   
    \PY{c+c1}{\PYZsh{} Zapisywanie animacji do plików}
    \PY{n}{anim}\PY{o}{.}\PY{n}{save}\PY{p}{(}\PY{l+s+s1}{\PYZsq{}}\PY{l+s+s1}{sphere\PYZus{}3d.gif}\PY{l+s+s1}{\PYZsq{}}\PY{p}{,} \PY{n}{writer}\PY{o}{=}\PY{l+s+s1}{\PYZsq{}}\PY{l+s+s1}{imagemagick}\PY{l+s+s1}{\PYZsq{}}\PY{p}{)}
    \PY{n}{anim\PYZus{}2d}\PY{o}{.}\PY{n}{save}\PY{p}{(}\PY{l+s+s1}{\PYZsq{}}\PY{l+s+s1}{sphere.gif}\PY{l+s+s1}{\PYZsq{}}\PY{p}{,} \PY{n}{writer}\PY{o}{=}\PY{l+s+s1}{\PYZsq{}}\PY{l+s+s1}{imagemagick}\PY{l+s+s1}{\PYZsq{}}\PY{p}{)}
   
    \PY{n}{plt}\PY{o}{.}\PY{n}{show}\PY{p}{(}\PY{p}{)}


\PY{c+c1}{\PYZsh{} Tworzenie populacji mrówek}
\PY{n}{num\PYZus{}ants} \PY{o}{=} \PY{l+m+mi}{100}
\PY{n}{ants} \PY{o}{=} \PY{p}{[}\PY{n}{Ant}\PY{p}{(}\PY{n}{np}\PY{o}{.}\PY{n}{random}\PY{o}{.}\PY{n}{uniform}\PY{p}{(}\PY{o}{\PYZhy{}}\PY{l+m+mi}{10}\PY{p}{,} \PY{l+m+mi}{10}\PY{p}{)}\PY{p}{,} \PY{n}{np}\PY{o}{.}\PY{n}{random}\PY{o}{.}\PY{n}{uniform}\PY{p}{(}\PY{o}{\PYZhy{}}\PY{l+m+mi}{10}\PY{p}{,} \PY{l+m+mi}{10}\PY{p}{)}\PY{p}{)} \PY{k}{for} \PY{n}{\PYZus{}} \PY{o+ow}{in} \PY{n+nb}{range}\PY{p}{(}\PY{n}{num\PYZus{}ants}\PY{p}{)}\PY{p}{]}


\PY{c+c1}{\PYZsh{} Uruchomienie animacji}
\PY{n}{num\PYZus{}steps} \PY{o}{=} \PY{l+m+mi}{100}
\PY{n}{step\PYZus{}size} \PY{o}{=} \PY{l+m+mf}{0.1}
\PY{n}{plot\PYZus{}animation}\PY{p}{(}\PY{n}{ants}\PY{p}{,} \PY{n}{num\PYZus{}steps}\PY{p}{,} \PY{n}{step\PYZus{}size}\PY{p}{)}
\end{Verbatim}
\end{tcolorbox}

    \hypertarget{wykresy-dla-funkcji-celu---rastrigin}{%
\subsection{4.6 Wykresy dla Funkcji celu -
Rastrigin}\label{wykresy-dla-funkcji-celu---rastrigin}}

Funkcja plot\_animation służy do generowania animacji wykresów 2D i 3D
dla algorytmu mrówkowego, korzystając z danych populacji mrówek, liczby
kroków i rozmiaru kroku. Tworzone są dwa wykresy: jeden w trzech
wymiarach (fig\_3d) i drugi w dwóch wymiarach (fig\_2d).

Na koniec, funkcje animation.FuncAnimation tworzą animacje na podstawie
figur i funkcji update dla wykresów 3D i 2D. Animacje są zapisywane do
plików GIF (rastrigin\_3d.gif i rastrigin.gif). Następnie wyświetlane są
wykresy.

    \begin{tcolorbox}[breakable, size=fbox, boxrule=1pt, pad at break*=1mm,colback=cellbackground, colframe=cellborder]
\prompt{In}{incolor}{ }{\boxspacing}
\begin{Verbatim}[commandchars=\\\{\}]
\PY{k}{def} \PY{n+nf}{plot\PYZus{}animation}\PY{p}{(}\PY{n}{ants}\PY{p}{,} \PY{n}{num\PYZus{}steps}\PY{p}{,} \PY{n}{step\PYZus{}size}\PY{p}{)}\PY{p}{:}
    \PY{n}{fig\PYZus{}3d} \PY{o}{=} \PY{n}{plt}\PY{o}{.}\PY{n}{figure}\PY{p}{(}\PY{p}{)}
    \PY{n}{ax\PYZus{}3d} \PY{o}{=} \PY{n}{fig\PYZus{}3d}\PY{o}{.}\PY{n}{add\PYZus{}subplot}\PY{p}{(}\PY{l+m+mi}{111}\PY{p}{,} \PY{n}{projection}\PY{o}{=}\PY{l+s+s1}{\PYZsq{}}\PY{l+s+s1}{3d}\PY{l+s+s1}{\PYZsq{}}\PY{p}{)}


    \PY{n}{fig\PYZus{}2d} \PY{o}{=} \PY{n}{plt}\PY{o}{.}\PY{n}{figure}\PY{p}{(}\PY{p}{)}
    \PY{n}{ax\PYZus{}2d} \PY{o}{=} \PY{n}{fig\PYZus{}2d}\PY{o}{.}\PY{n}{add\PYZus{}subplot}\PY{p}{(}\PY{l+m+mi}{111}\PY{p}{)}


    \PY{n}{x\PYZus{}range} \PY{o}{=} \PY{n}{np}\PY{o}{.}\PY{n}{linspace}\PY{p}{(}\PY{o}{\PYZhy{}}\PY{l+m+mi}{10}\PY{p}{,} \PY{l+m+mi}{10}\PY{p}{,} \PY{l+m+mi}{200}\PY{p}{)}  \PY{c+c1}{\PYZsh{} Zwiększenie liczby próbek x}
    \PY{n}{y\PYZus{}range} \PY{o}{=} \PY{n}{np}\PY{o}{.}\PY{n}{linspace}\PY{p}{(}\PY{o}{\PYZhy{}}\PY{l+m+mi}{10}\PY{p}{,} \PY{l+m+mi}{10}\PY{p}{,} \PY{l+m+mi}{200}\PY{p}{)}  \PY{c+c1}{\PYZsh{} Zwiększenie liczby próbek y}
    \PY{n}{X}\PY{p}{,} \PY{n}{Y} \PY{o}{=} \PY{n}{np}\PY{o}{.}\PY{n}{meshgrid}\PY{p}{(}\PY{n}{x\PYZus{}range}\PY{p}{,} \PY{n}{y\PYZus{}range}\PY{p}{)}
    \PY{n}{Z} \PY{o}{=} \PY{n}{rastrigin\PYZus{}function}\PY{p}{(}\PY{n}{X}\PY{p}{,} \PY{n}{Y}\PY{p}{)}


    \PY{n}{ax\PYZus{}3d}\PY{o}{.}\PY{n}{plot\PYZus{}surface}\PY{p}{(}\PY{n}{X}\PY{p}{,} \PY{n}{Y}\PY{p}{,} \PY{n}{Z}\PY{p}{,} \PY{n}{cmap}\PY{o}{=}\PY{l+s+s1}{\PYZsq{}}\PY{l+s+s1}{viridis}\PY{l+s+s1}{\PYZsq{}}\PY{p}{,} \PY{n}{alpha}\PY{o}{=}\PY{l+m+mf}{0.8}\PY{p}{)}


    \PY{n}{ant\PYZus{}points} \PY{o}{=} \PY{n}{ax\PYZus{}3d}\PY{o}{.}\PY{n}{scatter}\PY{p}{(}\PY{p}{[}\PY{n}{ant}\PY{o}{.}\PY{n}{x} \PY{k}{for} \PY{n}{ant} \PY{o+ow}{in} \PY{n}{ants}\PY{p}{]}\PY{p}{,} \PY{p}{[}\PY{n}{ant}\PY{o}{.}\PY{n}{y} \PY{k}{for} \PY{n}{ant} \PY{o+ow}{in} \PY{n}{ants}\PY{p}{]}\PY{p}{,} \PY{p}{[}\PY{n}{ant}\PY{o}{.}\PY{n}{z} \PY{k}{for} \PY{n}{ant} \PY{o+ow}{in} \PY{n}{ants}\PY{p}{]}\PY{p}{,} \PY{n}{color}\PY{o}{=}\PY{l+s+s1}{\PYZsq{}}\PY{l+s+s1}{blue}\PY{l+s+s1}{\PYZsq{}}\PY{p}{)}
    \PY{n}{best\PYZus{}ant} \PY{o}{=} \PY{n+nb}{min}\PY{p}{(}\PY{n}{ants}\PY{p}{,} \PY{n}{key}\PY{o}{=}\PY{k}{lambda} \PY{n}{ant}\PY{p}{:} \PY{n}{ant}\PY{o}{.}\PY{n}{z}\PY{p}{)}
    \PY{n}{best\PYZus{}point} \PY{o}{=} \PY{n}{ax\PYZus{}3d}\PY{o}{.}\PY{n}{scatter}\PY{p}{(}\PY{n}{best\PYZus{}ant}\PY{o}{.}\PY{n}{x}\PY{p}{,} \PY{n}{best\PYZus{}ant}\PY{o}{.}\PY{n}{y}\PY{p}{,} \PY{n}{best\PYZus{}ant}\PY{o}{.}\PY{n}{z}\PY{p}{,} \PY{n}{color}\PY{o}{=}\PY{l+s+s1}{\PYZsq{}}\PY{l+s+s1}{red}\PY{l+s+s1}{\PYZsq{}}\PY{p}{)}
    \PY{n}{iter\PYZus{}text\PYZus{}3d} \PY{o}{=} \PY{n}{ax\PYZus{}3d}\PY{o}{.}\PY{n}{text2D}\PY{p}{(}\PY{l+m+mf}{0.05}\PY{p}{,} \PY{l+m+mf}{0.95}\PY{p}{,} \PY{l+s+s2}{\PYZdq{}}\PY{l+s+s2}{\PYZdq{}}\PY{p}{,} \PY{n}{transform}\PY{o}{=}\PY{n}{ax\PYZus{}3d}\PY{o}{.}\PY{n}{transAxes}\PY{p}{)}
    \PY{n}{best\PYZus{}text\PYZus{}3d} \PY{o}{=} \PY{n}{ax\PYZus{}3d}\PY{o}{.}\PY{n}{text2D}\PY{p}{(}\PY{l+m+mf}{0.05}\PY{p}{,} \PY{l+m+mf}{0.90}\PY{p}{,} \PY{l+s+s2}{\PYZdq{}}\PY{l+s+s2}{\PYZdq{}}\PY{p}{,} \PY{n}{transform}\PY{o}{=}\PY{n}{ax\PYZus{}3d}\PY{o}{.}\PY{n}{transAxes}\PY{p}{)}


    \PY{n}{ax\PYZus{}3d}\PY{o}{.}\PY{n}{set\PYZus{}xlabel}\PY{p}{(}\PY{l+s+s1}{\PYZsq{}}\PY{l+s+s1}{X}\PY{l+s+s1}{\PYZsq{}}\PY{p}{)}
    \PY{n}{ax\PYZus{}3d}\PY{o}{.}\PY{n}{set\PYZus{}ylabel}\PY{p}{(}\PY{l+s+s1}{\PYZsq{}}\PY{l+s+s1}{Y}\PY{l+s+s1}{\PYZsq{}}\PY{p}{)}
    \PY{n}{ax\PYZus{}3d}\PY{o}{.}\PY{n}{set\PYZus{}zlabel}\PY{p}{(}\PY{l+s+s1}{\PYZsq{}}\PY{l+s+s1}{Z}\PY{l+s+s1}{\PYZsq{}}\PY{p}{)}
    \PY{n}{ax\PYZus{}3d}\PY{o}{.}\PY{n}{set\PYZus{}title}\PY{p}{(}\PY{l+s+s1}{\PYZsq{}}\PY{l+s+s1}{Algorytm mrówkowy \PYZhy{} funkcja Rastrigin}\PY{l+s+s1}{\PYZsq{}}\PY{p}{)}


    \PY{n}{iter\PYZus{}text\PYZus{}2d} \PY{o}{=} \PY{n}{ax\PYZus{}2d}\PY{o}{.}\PY{n}{text}\PY{p}{(}\PY{l+m+mf}{0.05}\PY{p}{,} \PY{l+m+mf}{0.95}\PY{p}{,} \PY{l+s+s2}{\PYZdq{}}\PY{l+s+s2}{\PYZdq{}}\PY{p}{,} \PY{n}{transform}\PY{o}{=}\PY{n}{ax\PYZus{}2d}\PY{o}{.}\PY{n}{transAxes}\PY{p}{)}
    \PY{n}{best\PYZus{}text\PYZus{}2d} \PY{o}{=} \PY{n}{ax\PYZus{}2d}\PY{o}{.}\PY{n}{text}\PY{p}{(}\PY{l+m+mf}{0.05}\PY{p}{,} \PY{l+m+mf}{0.90}\PY{p}{,} \PY{l+s+s2}{\PYZdq{}}\PY{l+s+s2}{\PYZdq{}}\PY{p}{,} \PY{n}{transform}\PY{o}{=}\PY{n}{ax\PYZus{}2d}\PY{o}{.}\PY{n}{transAxes}\PY{p}{)}


    \PY{k}{def} \PY{n+nf}{update}\PY{p}{(}\PY{n}{frame}\PY{p}{)}\PY{p}{:}
        \PY{k}{for} \PY{n}{ant} \PY{o+ow}{in} \PY{n}{ants}\PY{p}{:}
            \PY{n}{update\PYZus{}ant}\PY{p}{(}\PY{n}{ant}\PY{p}{,} \PY{n}{step\PYZus{}size}\PY{p}{)}


        \PY{n}{ant\PYZus{}points}\PY{o}{.}\PY{n}{\PYZus{}offsets3d} \PY{o}{=} \PY{p}{(}\PY{p}{[}\PY{n}{ant}\PY{o}{.}\PY{n}{x} \PY{k}{for} \PY{n}{ant} \PY{o+ow}{in} \PY{n}{ants}\PY{p}{]}\PY{p}{,} \PY{p}{[}\PY{n}{ant}\PY{o}{.}\PY{n}{y} \PY{k}{for} \PY{n}{ant} \PY{o+ow}{in} \PY{n}{ants}\PY{p}{]}\PY{p}{,} \PY{p}{[}\PY{n}{ant}\PY{o}{.}\PY{n}{z} \PY{k}{for} \PY{n}{ant} \PY{o+ow}{in} \PY{n}{ants}\PY{p}{]}\PY{p}{)}
        \PY{n}{best\PYZus{}ant} \PY{o}{=} \PY{n+nb}{min}\PY{p}{(}\PY{n}{ants}\PY{p}{,} \PY{n}{key}\PY{o}{=}\PY{k}{lambda} \PY{n}{ant}\PY{p}{:} \PY{n}{ant}\PY{o}{.}\PY{n}{z}\PY{p}{)}
        \PY{n}{best\PYZus{}point}\PY{o}{.}\PY{n}{\PYZus{}offsets3d} \PY{o}{=} \PY{p}{(}\PY{p}{[}\PY{n}{best\PYZus{}ant}\PY{o}{.}\PY{n}{x}\PY{p}{]}\PY{p}{,} \PY{p}{[}\PY{n}{best\PYZus{}ant}\PY{o}{.}\PY{n}{y}\PY{p}{]}\PY{p}{,} \PY{p}{[}\PY{n}{best\PYZus{}ant}\PY{o}{.}\PY{n}{z}\PY{p}{]}\PY{p}{)}
        \PY{n}{iter\PYZus{}text\PYZus{}3d}\PY{o}{.}\PY{n}{set\PYZus{}text}\PY{p}{(}\PY{l+s+sa}{f}\PY{l+s+s2}{\PYZdq{}}\PY{l+s+s2}{Iteracja: }\PY{l+s+si}{\PYZob{}}\PY{n}{frame}\PY{o}{+}\PY{l+m+mi}{1}\PY{l+s+si}{\PYZcb{}}\PY{l+s+s2}{\PYZdq{}}\PY{p}{)}
        \PY{n}{best\PYZus{}text\PYZus{}3d}\PY{o}{.}\PY{n}{set\PYZus{}text}\PY{p}{(}\PY{l+s+sa}{f}\PY{l+s+s2}{\PYZdq{}}\PY{l+s+s2}{Najlepszy wynik: }\PY{l+s+si}{\PYZob{}}\PY{n}{best\PYZus{}ant}\PY{o}{.}\PY{n}{z}\PY{l+s+si}{:}\PY{l+s+s2}{.7f}\PY{l+s+si}{\PYZcb{}}\PY{l+s+s2}{\PYZdq{}}\PY{p}{)}


        \PY{n}{ax\PYZus{}2d}\PY{o}{.}\PY{n}{clear}\PY{p}{(}\PY{p}{)}
        \PY{n}{ax\PYZus{}2d}\PY{o}{.}\PY{n}{contourf}\PY{p}{(}\PY{n}{X}\PY{p}{,} \PY{n}{Y}\PY{p}{,} \PY{n}{Z}\PY{p}{,} \PY{n}{cmap}\PY{o}{=}\PY{l+s+s1}{\PYZsq{}}\PY{l+s+s1}{viridis}\PY{l+s+s1}{\PYZsq{}}\PY{p}{,} \PY{n}{alpha}\PY{o}{=}\PY{l+m+mf}{0.8}\PY{p}{)}
        \PY{n}{ax\PYZus{}2d}\PY{o}{.}\PY{n}{plot}\PY{p}{(}\PY{p}{[}\PY{n}{ant}\PY{o}{.}\PY{n}{x} \PY{k}{for} \PY{n}{ant} \PY{o+ow}{in} \PY{n}{ants}\PY{p}{]}\PY{p}{,} \PY{p}{[}\PY{n}{ant}\PY{o}{.}\PY{n}{y} \PY{k}{for} \PY{n}{ant} \PY{o+ow}{in} \PY{n}{ants}\PY{p}{]}\PY{p}{,} \PY{l+s+s1}{\PYZsq{}}\PY{l+s+s1}{bo}\PY{l+s+s1}{\PYZsq{}}\PY{p}{)}
        \PY{n}{ax\PYZus{}2d}\PY{o}{.}\PY{n}{plot}\PY{p}{(}\PY{n}{best\PYZus{}ant}\PY{o}{.}\PY{n}{x}\PY{p}{,} \PY{n}{best\PYZus{}ant}\PY{o}{.}\PY{n}{y}\PY{p}{,} \PY{l+s+s1}{\PYZsq{}}\PY{l+s+s1}{ro}\PY{l+s+s1}{\PYZsq{}}\PY{p}{)}
        \PY{n}{iter\PYZus{}text\PYZus{}2d}\PY{o}{.}\PY{n}{set\PYZus{}text}\PY{p}{(}\PY{l+s+sa}{f}\PY{l+s+s2}{\PYZdq{}}\PY{l+s+s2}{Iteracja: }\PY{l+s+si}{\PYZob{}}\PY{n}{frame}\PY{o}{+}\PY{l+m+mi}{1}\PY{l+s+si}{\PYZcb{}}\PY{l+s+s2}{\PYZdq{}}\PY{p}{)}
        \PY{n}{best\PYZus{}text\PYZus{}2d}\PY{o}{.}\PY{n}{set\PYZus{}text}\PY{p}{(}\PY{l+s+sa}{f}\PY{l+s+s2}{\PYZdq{}}\PY{l+s+s2}{Najlepszy wynik: }\PY{l+s+si}{\PYZob{}}\PY{n}{best\PYZus{}ant}\PY{o}{.}\PY{n}{z}\PY{l+s+si}{:}\PY{l+s+s2}{.7f}\PY{l+s+si}{\PYZcb{}}\PY{l+s+s2}{\PYZdq{}}\PY{p}{)}


    \PY{n}{anim} \PY{o}{=} \PY{n}{animation}\PY{o}{.}\PY{n}{FuncAnimation}\PY{p}{(}\PY{n}{fig\PYZus{}3d}\PY{p}{,} \PY{n}{update}\PY{p}{,} \PY{n}{frames}\PY{o}{=}\PY{n}{num\PYZus{}steps}\PY{p}{,} \PY{n}{interval}\PY{o}{=}\PY{l+m+mi}{200}\PY{p}{,} \PY{n}{repeat}\PY{o}{=}\PY{k+kc}{False}\PY{p}{)}
    \PY{n}{anim\PYZus{}2d} \PY{o}{=} \PY{n}{animation}\PY{o}{.}\PY{n}{FuncAnimation}\PY{p}{(}\PY{n}{fig\PYZus{}2d}\PY{p}{,} \PY{n}{update}\PY{p}{,} \PY{n}{frames}\PY{o}{=}\PY{n}{num\PYZus{}steps}\PY{p}{,} \PY{n}{interval}\PY{o}{=}\PY{l+m+mi}{200}\PY{p}{,} \PY{n}{repeat}\PY{o}{=}\PY{k+kc}{False}\PY{p}{)}


    \PY{c+c1}{\PYZsh{} Zapisywanie animacji do plików}
    \PY{n}{anim}\PY{o}{.}\PY{n}{save}\PY{p}{(}\PY{l+s+s1}{\PYZsq{}}\PY{l+s+s1}{rastrigin\PYZus{}3d.gif}\PY{l+s+s1}{\PYZsq{}}\PY{p}{,} \PY{n}{writer}\PY{o}{=}\PY{l+s+s1}{\PYZsq{}}\PY{l+s+s1}{imagemagick}\PY{l+s+s1}{\PYZsq{}}\PY{p}{)}
    \PY{n}{anim\PYZus{}2d}\PY{o}{.}\PY{n}{save}\PY{p}{(}\PY{l+s+s1}{\PYZsq{}}\PY{l+s+s1}{rastrigin.gif}\PY{l+s+s1}{\PYZsq{}}\PY{p}{,} \PY{n}{writer}\PY{o}{=}\PY{l+s+s1}{\PYZsq{}}\PY{l+s+s1}{imagemagick}\PY{l+s+s1}{\PYZsq{}}\PY{p}{)}


    \PY{n}{plt}\PY{o}{.}\PY{n}{show}\PY{p}{(}\PY{p}{)}



\PY{c+c1}{\PYZsh{} Tworzenie populacji mrówek}
\PY{n}{num\PYZus{}ants} \PY{o}{=} \PY{l+m+mi}{100}
\PY{n}{ants} \PY{o}{=} \PY{p}{[}\PY{n}{Ant}\PY{p}{(}\PY{n}{np}\PY{o}{.}\PY{n}{random}\PY{o}{.}\PY{n}{uniform}\PY{p}{(}\PY{o}{\PYZhy{}}\PY{l+m+mi}{10}\PY{p}{,} \PY{l+m+mi}{10}\PY{p}{)}\PY{p}{,} \PY{n}{np}\PY{o}{.}\PY{n}{random}\PY{o}{.}\PY{n}{uniform}\PY{p}{(}\PY{o}{\PYZhy{}}\PY{l+m+mi}{10}\PY{p}{,} \PY{l+m+mi}{10}\PY{p}{)}\PY{p}{)} \PY{k}{for} \PY{n}{\PYZus{}} \PY{o+ow}{in} \PY{n+nb}{range}\PY{p}{(}\PY{n}{num\PYZus{}ants}\PY{p}{)}\PY{p}{]}



\PY{c+c1}{\PYZsh{} Uruchomienie animacji}
\PY{n}{num\PYZus{}steps} \PY{o}{=} \PY{l+m+mi}{100}
\PY{n}{step\PYZus{}size} \PY{o}{=} \PY{l+m+mf}{0.1}
\PY{n}{plot\PYZus{}animation}\PY{p}{(}\PY{n}{ants}\PY{p}{,} \PY{n}{num\PYZus{}steps}\PY{p}{,} \PY{n}{step\PYZus{}size}\PY{p}{)}
\end{Verbatim}
\end{tcolorbox}

    \hypertarget{wykresy-przedstawiajux105-zmianux119-najlepszego-wyniku-dla-kaux17cdej-funkcji-w-kolejnych-iteracjach}{%
\subsection{4.7 Wykresy przedstawiają zmianę najlepszego wyniku dla
każdej funkcji w kolejnych
iteracjach}\label{wykresy-przedstawiajux105-zmianux119-najlepszego-wyniku-dla-kaux17cdej-funkcji-w-kolejnych-iteracjach}}

    \begin{tcolorbox}[breakable, size=fbox, boxrule=1pt, pad at break*=1mm,colback=cellbackground, colframe=cellborder]
\prompt{In}{incolor}{ }{\boxspacing}
\begin{Verbatim}[commandchars=\\\{\}]
\PY{k}{for} \PY{n}{it} \PY{o+ow}{in} \PY{n+nb}{range}\PY{p}{(}\PY{n}{n\PYZus{}iterations}\PY{p}{)}\PY{p}{:}
    \PY{c+c1}{\PYZsh{} Każda mrówka wykonuje ruch}
    \PY{k}{for} \PY{n}{i} \PY{o+ow}{in} \PY{n+nb}{range}\PY{p}{(}\PY{n}{n\PYZus{}ants}\PY{p}{)}\PY{p}{:}
        \PY{c+c1}{\PYZsh{} Losowo wybiera kierunek}
        \PY{n}{direction} \PY{o}{=} \PY{n}{np}\PY{o}{.}\PY{n}{random}\PY{o}{.}\PY{n}{uniform}\PY{p}{(}\PY{o}{\PYZhy{}}\PY{l+m+mi}{1}\PY{p}{,} \PY{l+m+mi}{1}\PY{p}{,} \PY{l+m+mi}{2}\PY{p}{)}
        \PY{n}{direction} \PY{o}{/}\PY{o}{=} \PY{n}{np}\PY{o}{.}\PY{n}{linalg}\PY{o}{.}\PY{n}{norm}\PY{p}{(}\PY{n}{direction}\PY{p}{)} \PY{c+c1}{\PYZsh{} Normalize to unit vector}


        \PY{c+c1}{\PYZsh{} Zaktualizuj pozycje}
        \PY{n}{ants}\PY{p}{[}\PY{n}{i}\PY{p}{]} \PY{o}{+}\PY{o}{=} \PY{n}{step\PYZus{}size} \PY{o}{*} \PY{n}{direction}


        \PY{c+c1}{\PYZsh{} Kontrola granicy}
        \PY{n}{ants}\PY{p}{[}\PY{n}{i}\PY{p}{]} \PY{o}{=} \PY{n}{np}\PY{o}{.}\PY{n}{clip}\PY{p}{(}\PY{n}{ants}\PY{p}{[}\PY{n}{i}\PY{p}{]}\PY{p}{,} \PY{o}{\PYZhy{}}\PY{l+m+mi}{10}\PY{p}{,} \PY{l+m+mi}{10}\PY{p}{)}


        \PY{c+c1}{\PYZsh{} Zaktualizuj najlepsze rozwiązanie dla funkcji celu \PYZhy{} sfera}
        \PY{n}{score\PYZus{}sphere} \PY{o}{=} \PY{n}{sphere\PYZus{}function}\PY{p}{(}\PY{n}{ants}\PY{p}{[}\PY{n}{i}\PY{p}{]}\PY{p}{[}\PY{l+m+mi}{0}\PY{p}{]}\PY{p}{,} \PY{n}{ants}\PY{p}{[}\PY{n}{i}\PY{p}{]}\PY{p}{[}\PY{l+m+mi}{1}\PY{p}{]}\PY{p}{)}
        \PY{k}{if} \PY{n}{score\PYZus{}sphere} \PY{o}{\PYZlt{}} \PY{n}{best\PYZus{}score\PYZus{}sphere}\PY{p}{:}
            \PY{n}{best\PYZus{}score\PYZus{}sphere} \PY{o}{=} \PY{n}{score\PYZus{}sphere}
            \PY{n}{best\PYZus{}ant\PYZus{}sphere} \PY{o}{=} \PY{n}{ants}\PY{p}{[}\PY{n}{i}\PY{p}{]}\PY{o}{.}\PY{n}{copy}\PY{p}{(}\PY{p}{)}


        \PY{c+c1}{\PYZsh{} Zaktualizuj najlepsze rozwiązanie dla funkcji celu \PYZhy{} Rastrigin}
        \PY{n}{score\PYZus{}rastrigin} \PY{o}{=} \PY{n}{rastrigin\PYZus{}function}\PY{p}{(}\PY{n}{ants}\PY{p}{[}\PY{n}{i}\PY{p}{]}\PY{p}{[}\PY{l+m+mi}{0}\PY{p}{]}\PY{p}{,} \PY{n}{ants}\PY{p}{[}\PY{n}{i}\PY{p}{]}\PY{p}{[}\PY{l+m+mi}{1}\PY{p}{]}\PY{p}{)}
        \PY{k}{if} \PY{n}{score\PYZus{}rastrigin} \PY{o}{\PYZlt{}} \PY{n}{best\PYZus{}score\PYZus{}rastrigin}\PY{p}{:}
            \PY{n}{best\PYZus{}score\PYZus{}rastrigin} \PY{o}{=} \PY{n}{score\PYZus{}rastrigin}
            \PY{n}{best\PYZus{}ant\PYZus{}rastrigin} \PY{o}{=} \PY{n}{ants}\PY{p}{[}\PY{n}{i}\PY{p}{]}\PY{o}{.}\PY{n}{copy}\PY{p}{(}\PY{p}{)}


    \PY{n}{best\PYZus{}scores\PYZus{}sphere}\PY{o}{.}\PY{n}{append}\PY{p}{(}\PY{n}{best\PYZus{}score\PYZus{}sphere}\PY{p}{)}
    \PY{n}{best\PYZus{}scores\PYZus{}rastrigin}\PY{o}{.}\PY{n}{append}\PY{p}{(}\PY{n}{best\PYZus{}score\PYZus{}rastrigin}\PY{p}{)}
\end{Verbatim}
\end{tcolorbox}

    \hypertarget{wykresu-funkcji-celu---sfera}{%
\subsubsection{Wykresu funkcji celu -
sfera}\label{wykresu-funkcji-celu---sfera}}

Tworzymy nowe okno dla wykresu funkcji sfera za pomocą plt.figure().
Następnie rysujemy wykres najlepszego wyniku dla każdej iteracji.
Dodajemy etykiety osi i tytuł wykresu. Na koniec używamy plt.savefig()
do zapisania wykresu do pliku ``sphere.png''.

    \begin{tcolorbox}[breakable, size=fbox, boxrule=1pt, pad at break*=1mm,colback=cellbackground, colframe=cellborder]
\prompt{In}{incolor}{ }{\boxspacing}
\begin{Verbatim}[commandchars=\\\{\}]
\PY{c+c1}{\PYZsh{} Wykres zbieżności funkcji celu \PYZhy{} sfera}
\PY{n}{plt}\PY{o}{.}\PY{n}{figure}\PY{p}{(}\PY{p}{)}
\PY{n}{plt}\PY{o}{.}\PY{n}{plot}\PY{p}{(}\PY{n}{best\PYZus{}scores\PYZus{}sphere}\PY{p}{)}
\PY{n}{plt}\PY{o}{.}\PY{n}{xlabel}\PY{p}{(}\PY{l+s+s1}{\PYZsq{}}\PY{l+s+s1}{Iteration}\PY{l+s+s1}{\PYZsq{}}\PY{p}{)}
\PY{n}{plt}\PY{o}{.}\PY{n}{ylabel}\PY{p}{(}\PY{l+s+s1}{\PYZsq{}}\PY{l+s+s1}{Best Score}\PY{l+s+s1}{\PYZsq{}}\PY{p}{)}
\PY{n}{plt}\PY{o}{.}\PY{n}{title}\PY{p}{(}\PY{l+s+s1}{\PYZsq{}}\PY{l+s+s1}{Convergence \PYZhy{} Sphere Function}\PY{l+s+s1}{\PYZsq{}}\PY{p}{)}
\PY{n}{plt}\PY{o}{.}\PY{n}{savefig}\PY{p}{(}\PY{l+s+s1}{\PYZsq{}}\PY{l+s+s1}{sphere.png}\PY{l+s+s1}{\PYZsq{}}\PY{p}{)}
\end{Verbatim}
\end{tcolorbox}

    \hypertarget{wykresu-funkcji-celu---rastrigina}{%
\subsubsection{Wykresu funkcji celu -
Rastrigina}\label{wykresu-funkcji-celu---rastrigina}}

Analogicznie tworzymy nowe okno dla wykresu funkcji Rastrigina i
zapisujemy go do pliku ``rastrigin.png'' przy użyciu plt.savefig().

    \begin{tcolorbox}[breakable, size=fbox, boxrule=1pt, pad at break*=1mm,colback=cellbackground, colframe=cellborder]
\prompt{In}{incolor}{ }{\boxspacing}
\begin{Verbatim}[commandchars=\\\{\}]
\PY{c+c1}{\PYZsh{} Wykres zbieżności funkcji celu \PYZhy{} Rastrigin}
\PY{n}{plt}\PY{o}{.}\PY{n}{figure}\PY{p}{(}\PY{p}{)}
\PY{n}{plt}\PY{o}{.}\PY{n}{plot}\PY{p}{(}\PY{n}{best\PYZus{}scores\PYZus{}rastrigin}\PY{p}{)}
\PY{n}{plt}\PY{o}{.}\PY{n}{xlabel}\PY{p}{(}\PY{l+s+s1}{\PYZsq{}}\PY{l+s+s1}{Iteration}\PY{l+s+s1}{\PYZsq{}}\PY{p}{)}
\PY{n}{plt}\PY{o}{.}\PY{n}{ylabel}\PY{p}{(}\PY{l+s+s1}{\PYZsq{}}\PY{l+s+s1}{Best Score}\PY{l+s+s1}{\PYZsq{}}\PY{p}{)}
\PY{n}{plt}\PY{o}{.}\PY{n}{title}\PY{p}{(}\PY{l+s+s1}{\PYZsq{}}\PY{l+s+s1}{Convergence \PYZhy{} Rastrigin Function}\PY{l+s+s1}{\PYZsq{}}\PY{p}{)}
\PY{n}{plt}\PY{o}{.}\PY{n}{savefig}\PY{p}{(}\PY{l+s+s1}{\PYZsq{}}\PY{l+s+s1}{rastrigin.png}\PY{l+s+s1}{\PYZsq{}}\PY{p}{)}
\end{Verbatim}
\end{tcolorbox}

    \hypertarget{podsumowanie}{%
\subsection{Podsumowanie:}\label{podsumowanie}}

Algorytmy genetyczne są zaawansowanymi narzędziami optymalizacyjnymi,
które naśladują procesy ewolucyjne zachodzące w przyrodzie. Opierając
się na pojęciach takich jak selekcja naturalna, krzyżowanie i mutacja,
algorytmy genetyczne są w stanie generować i doskonalić populacje
potencjalnych rozwiązań w celu znalezienia optymalnego rozwiązania dla
różnorodnych problemów.

W trakcie działania algorytmu genetycznego, początkowa populacja
rozwiązań jest losowo generowana. Następnie przeprowadza się operacje
selekcji, krzyżowania i mutacji, które wpływają na ewolucję populacji w
kierunku lepszych rozwiązań. Proces ten jest powtarzany przez określoną
liczbę generacji lub do spełnienia warunku stopu, takiego jak
osiągnięcie zadowalającego rozwiązania.

Zaletą algorytmów genetycznych jest ich zdolność do radzenia sobie z
problemami optymalizacyjnymi o dużym stopniu złożoności, w tym takimi,
dla których brakuje efektywnych metod rozwiązywania. Mogą być stosowane
w różnych dziedzinach, takich jak inżynieria, ekonomia, biologia,
logistyka czy planowanie. Algorytmy genetyczne są szczególnie przydatne
w przypadkach, gdy problem ma wiele zmiennych decyzyjnych i brak jest
jasno określonych reguł dotyczących optymalnego rozwiązania.

\hypertarget{wnioski}{%
\subsection{Wnioski:}\label{wnioski}}

Po przeprowadzeniu analizy algorytmów genetycznych oraz ich zastosowań,
można wyciągnąć kilka istotnych wniosków:

\begin{enumerate}
\def\labelenumi{\arabic{enumi}.}
\item
  Skuteczność: Algorytmy genetyczne mogą być bardzo skuteczne w
  rozwiązywaniu problemów optymalizacyjnych. Dzięki swojej zdolności do
  przeszukiwania przestrzeni rozwiązań i adaptacji do zmieniających się
  warunków, potrafią znaleźć dobre, a czasem nawet optymalne
  rozwiązania.
\item
  Zastosowanie: Algorytmy genetyczne znajdują szerokie zastosowanie w
  wielu dziedzinach, takich jak projektowanie układów, planowanie
  produkcji, analiza danych, optymalizacja tras czy tworzenie sztucznej
  inteligencji. Ich uniwersalność i skalowalność czynią je przydatnym
  narzędziem dla różnorodnych problemów.
\item
  Parametryzacja: Wybór odpowiednich parametrów algorytmu genetycznego,
  takich jak rozmiar populacji, prawdopodobieństwo mutacji czy operator
  selekcji, ma istotny wpływ na jego skuteczność. Optymalizacja tych
  parametrów może być czasochłonna i wymagać eksperymentów, ale jest
  kluczowa dla uzyskania najlepszych wyników.
\item
  Wyzwania: Algorytmy genetyczne również stawiają pewne wyzwania. Przy
  dużej ilości zmiennych decyzyjnych i rozległej przestrzeni rozwiązań,
  mogą wymagać dużego nakładu obliczeniowego. Ponadto, ryzyko utknięcia
  w lokalnym minimum jest również ważne, dlatego istnieje potrzeba
  stosowania zaawansowanych strategii, takich jak operator krzyżowania
  czy techniki redukcji błędów.
\end{enumerate}

Wnioskiem jest to, że algorytmy genetyczne stanowią ważne narzędzie
optymalizacyjne, które może być z powodzeniem stosowane w różnorodnych
problemach. Jednakże, ich skuteczność i efektywność są zależne od
właściwego doboru parametrów oraz zrozumienia specyfiki problemu. Dalsze
badania i rozwój algorytmów genetycznych są niezbędne, aby doskonalić
ich wydajność i dostosowywać je do coraz bardziej złożonych i
wymagających problemów.

    \hypertarget{podsumowanie}{%
\subsection{5. Podsumowanie:}\label{podsumowanie}}

Algorytmy genetyczne są zaawansowanymi narzędziami optymalizacyjnymi,
które naśladują procesy ewolucyjne zachodzące w przyrodzie. Opierając
się na pojęciach takich jak selekcja naturalna, krzyżowanie i mutacja,
algorytmy genetyczne są w stanie generować i doskonalić populacje
potencjalnych rozwiązań w celu znalezienia optymalnego rozwiązania dla
różnorodnych problemów.

W trakcie działania algorytmu genetycznego, początkowa populacja
rozwiązań jest losowo generowana. Następnie przeprowadza się operacje
selekcji, krzyżowania i mutacji, które wpływają na ewolucję populacji w
kierunku lepszych rozwiązań. Proces ten jest powtarzany przez określoną
liczbę generacji lub do spełnienia warunku stopu, takiego jak
osiągnięcie zadowalającego rozwiązania.

Zaletą algorytmów genetycznych jest ich zdolność do radzenia sobie z
problemami optymalizacyjnymi o dużym stopniu złożoności, w tym takimi,
dla których brakuje efektywnych metod rozwiązywania. Mogą być stosowane
w różnych dziedzinach, takich jak inżynieria, ekonomia, biologia,
logistyka czy planowanie. Algorytmy genetyczne są szczególnie przydatne
w przypadkach, gdy problem ma wiele zmiennych decyzyjnych i brak jest
jasno określonych reguł dotyczących optymalnego rozwiązania.

\hypertarget{wnioski}{%
\subsection{6. Wnioski:}\label{wnioski}}

Po przeprowadzeniu analizy algorytmów genetycznych oraz ich zastosowań,
można wyciągnąć kilka istotnych wniosków:

\begin{enumerate}
\def\labelenumi{\arabic{enumi}.}
\item
  Skuteczność: Algorytmy genetyczne mogą być bardzo skuteczne w
  rozwiązywaniu problemów optymalizacyjnych. Dzięki swojej zdolności do
  przeszukiwania przestrzeni rozwiązań i adaptacji do zmieniających się
  warunków, potrafią znaleźć dobre, a czasem nawet optymalne
  rozwiązania.
\item
  Zastosowanie: Algorytmy genetyczne znajdują szerokie zastosowanie w
  wielu dziedzinach, takich jak projektowanie układów, planowanie
  produkcji, analiza danych, optymalizacja tras czy tworzenie sztucznej
  inteligencji. Ich uniwersalność i skalowalność czynią je przydatnym
  narzędziem dla różnorodnych problemów.
\item
  Parametryzacja: Wybór odpowiednich parametrów algorytmu genetycznego,
  takich jak rozmiar populacji, prawdopodobieństwo mutacji czy operator
  selekcji, ma istotny wpływ na jego skuteczność. Optymalizacja tych
  parametrów może być czasochłonna i wymagać eksperymentów, ale jest
  kluczowa dla uzyskania najlepszych wyników.
\item
  Wyzwania: Algorytmy genetyczne również stawiają pewne wyzwania. Przy
  dużej ilości zmiennych decyzyjnych i rozległej przestrzeni rozwiązań,
  mogą wymagać dużego nakładu obliczeniowego. Ponadto, ryzyko utknięcia
  w lokalnym minimum jest również ważne, dlatego istnieje potrzeba
  stosowania zaawansowanych strategii, takich jak operator krzyżowania
  czy techniki redukcji błędów.
\end{enumerate}

Wnioskiem jest to, że algorytmy genetyczne stanowią ważne narzędzie
optymalizacyjne, które może być z powodzeniem stosowane w różnorodnych
problemach. Jednakże, ich skuteczność i efektywność są zależne od
właściwego doboru parametrów oraz zrozumienia specyfiki problemu. Dalsze
badania i rozwój algorytmów genetycznych są niezbędne, aby doskonalić
ich wydajność i dostosowywać je do coraz bardziej złożonych i
wymagających problemów.


    % Add a bibliography block to the postdoc
    
    
    
\end{document}
